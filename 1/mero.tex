\section{Analytic and Meromorphic Functions}
We have seen previously that there can be some multivalued functions in $\mathbb C$ carries important meanings.
Most important examples being $\log$ and $\sqrt[m]{\cdot}$.
What do we really mean by these multivalued functions?
When we evaluate them, we always have to choose a branch cut, but it does not capture the relationship between different branch cuts.
We shall try to find a way to characterise these functions in a proper way that realises their properties as they deserve.
\subsection{Analytic Functions and their Zeros}
\begin{definition}
    A domain is an open, connected subset $D\subset\mathbb C$.
\end{definition}
\begin{example}
    Open disks, annuli, punctured disks are domains.
\end{example}
\begin{definition}
    Let $D\subset\mathbb C$ be a domain.
    A function $f:D\to\mathbb C$ is holomorphic or analytic if either it is $\mathbb C$-differentiable everywhere on $D$ or it has a local Taylor series around every point in $D$.
\end{definition}
We know the two criteria above are equivalent from discussions back in IB Complex Analysis.
\begin{proposition}
    Let $f:D\to\mathbb C$ be an analytic functions on a domain.
    If $f(z_0)=0$, then either $f$ is identically zero or nowhere zero on a punctured disk centering at $z_0$.
\end{proposition}
\begin{proof}
    Obvious and already discussed in Complex Analysis, but let's do it again.
    Consider the Taylor series of $f$ in a neighbourhood $U$ of $z_0$.
    So for $z\in U$ we have
    $$f(z)=\sum_{n=0}^\infty a_n(z-z_0)^n$$
    If $f$ is not identically zero, then we can choose minimal $m$ such that $a_m\neq 0$, so $f(z)=(z-z_0)^mg(z)$ where
    $$g(z)=\sum_{n=0}^\infty a_{m+n}(z-z_0)^n\neq 0$$
    is nonzero at $z_0$, hence is nowhere zero in a punctured disk around $z_0$ in $U\subset D$ by continuity.
    $f$ is then nowhere zero in the same punctured disk.
\end{proof}
\begin{corollary}[Identity Principle]
    Let $f,g$ be analytic functions defined on a domain $D\subset\mathbb C$.
    If the subspace $\{z\in D|f(z)=g(z)\}$ is not discrete, then $f\equiv g$ on $D$.
\end{corollary}
\begin{proof}
    Immediate.
\end{proof}
\subsection{Meromorphic Functions and Singularities}
\begin{definition}
    A function $f$ defined on a punctured disk around $z_0$ is said to have a isolated singularity at $z_0$.
\end{definition}
\begin{proposition}
    If an analysic function $f$ has an isolated singularity at $z_0$, then $f$ has a Laurent series
    $$f(z)=\sum_{n=-\infty}^\infty a_n(z-z_0)^n$$
    on a punctured disk around $z_0$.
\end{proposition}
\begin{proof}
    Complex Analysis.
\end{proof}
\begin{definition}
    If $a_n=0$ for $n<0$, then $z_0$ is said to be a removable singularity of $f$.\\
    If $a_n=0$ for $n<-m<0$ and $a_{-m}\neq 0$, then we say $z_0$ is a pole of order $m$.\\
    Otherwise, we say $z_0$ is an essential singularity.
\end{definition}
\begin{theorem}
    If $f$ is bounded near $z_0$, then $z_0$ has to be a removable singularity.
\end{theorem}
\begin{proof}
    Complex Analysis.
\end{proof}
\begin{theorem}[Casorati-Weierstrass]
    $z_0$ is an essential singularity iff $f(U)$ is dense in $\mathbb C$ for any punctured neighbourhood of $z_0$.
\end{theorem}
\begin{proof}
    Complex Analysis.
\end{proof}
\begin{definition}
    If a holomorphic function $f:D\setminus A\to\mathbb C$ for some domain $D$ and discrete $A$ has poles at the points of $A$, then $f$ is said to be meromorphic.
\end{definition}
\begin{example}
    The function $f(z)=1/(e^{1/z}-1)$ is meromorphic where one takes $D$ to be the open upper half-plane and $A=\{1/(2\pi in):n\in\mathbb N\}$.
    In particular, the poles are all simple (i.e. of order $1$).\\
    Note that the function can be extended to the whole of $\mathbb C$ except at the discrete set $\{1/(2\pi in):n\in\mathbb Z\setminus\{0\}\}\cup\{0\}$, which contains $0$ as an essential singularity and others as simple poles.
\end{example}
\subsection{Analytic Continuation}
\begin{definition}
    A function element $F=(f,U)$ on a domain $D$ consists of a subdomain $U\subset D$ and an analytic function $f:U\to\mathbb C$.
\end{definition}
\begin{lemma}[Direct Analytic Continuation]
    Let $(f,U),(g,V)$ are function elements such that $U\cap V\neq\varnothing$ and $f=g$ on $U\cap V$, then $f$ determines $g$.
\end{lemma}
\begin{proof}
    Identity Principle.
\end{proof}
We write $(f,U)\sim (g,V)$ for direct analytic continuation.
\begin{definition}
    Analytic continuation is a sequence of iteration of direct analytic continuations that get from one function element to another.
    We write $(f,U)\approx (g,V)$ in this case.
\end{definition}
\begin{remark}
    $\approx$ is an equivalence relation.
\end{remark}
\begin{definition}
    A $\approx$-equivalence class $\mathcal F$ of function elements on a domain $D$ is called a complete analytic function on $D$.
\end{definition}
\subsection{The Complex Logarithm}
Let $\mathbb C_\star=\mathbb C\setminus\{0\}$.
We want to invert the exponential function $\exp:\mathbb C\to\mathbb C_\star$, but $\exp$ is not injective on $\mathbb C$.
We used to see $\log$ as a multivalued function to deal with this problem, but in fact, we can see it as a complete analytic function on $\mathbb C_\star$.\\
Indeed, for $(\alpha,\beta)\subset\mathbb R$ with $|\alpha-\beta|<2\pi$, we can define
$$U_{(\alpha,\beta)}=\{re^{i\theta}|r>0,\alpha<\theta<\beta\},f_{(\alpha,\beta)}(z)=\log r+i\theta,r=|z|,\theta\in (\alpha,\beta)$$
Then $F_{(\alpha,\beta)}=(f_{(\alpha,\beta)},U_{(\alpha,\beta)})$ is a collection of function elements.
Let $I(n)=((n-1)\pi/2,(n+1)\pi/2)$ for $n\in\mathbb Z$.
\begin{proposition}
    $F_{I(n)}\sim F_{I(m)}$ iff $|m-n|\le 1$.
\end{proposition}
\begin{proof}
    Just do a case analysis based on $m-n\bmod 4$.
\end{proof}
\begin{corollary}
    $F_{I(m)}\approx F_{I(n)}$ for any $m,n\in\mathbb Z$.
\end{corollary}
\begin{proof}
    Immediate.
\end{proof}
So this characterises a complete analytic function that is the complex logarithm.
\begin{remark}
    We know that $f_{I(0)}(1)=0$ but $f_{I(4)}(1)=2\pi i$, so analytic continuation in this way is not unique.
    But we have ``pasted'' them together to make it a complete analytic function.
\end{remark}
\begin{definition}
    Let $\gamma:[0,1]\to D$ be a path.
    We say $(f,U)\approx_\gamma (g,V)$ if there is some $0=t_0<t_1<\cdots<t_n=1$ and function elements $(f=f_0,U=U_0),(f_1,U_1),\ldots,(g=f_{n+1},V=U_{n+1})$ such that $(f_i,U_i)\sim (f_{i+1},U_{i+1})$ for $i=0,\ldots,n$ and $\gamma(t_i)\in U_i\cap U_{i+1}$.
    This is called analytic continuation along a path.
\end{definition}
Analytic continuation along a path has the desired uniqueness property.
This fact is known as the Classical Monodromy Theorem.