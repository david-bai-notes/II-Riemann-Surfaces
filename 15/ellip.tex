\section{More about Weierstrass' Elliptic Functions; Quotients}
\subsection{An Elliptic Curve}
We have seen two general constructions of Riemann surfaces with genus $1$:
The complex torus and the compactification of the Riemann surface associated with $w^2=z^3-z$.
These two constructions has to be related, but how?
Here, we shall prove that any complex torus is isomorphic to an algebraic construction.
This is a corollary of Proposition \ref{elliptic_wp}.
\begin{corollary}
    Let $\mathbb C/\Lambda$ be a complex torus, then there are constants $g_2,g_3$ such that $\mathbb C/\Lambda$ is biholomorphic to a one-point compactification of the graph $X'=\{(x,y)\in\mathbb C^2:y^2=4x^3-g_2x-g_3\}$.
\end{corollary}
\begin{proof}[Sketch of proof]
    Take $g_2,g_3$ exactly as in Proposition \ref{elliptic_wp}.
    Turns out $X'$ can be compactified into a Riemann surface $X=X'\cup\{\infty\}$ with charts provided by coordinate projection.
    Define $F:\mathbb C\to X$ via $z\mapsto(\wp(z),\wp^\prime(z))$ where $\wp=\wp_\Lambda$.
    Now $\operatorname{Im}F\subset X$ by Proposition \ref{elliptic_wp} and $F$ is analytic as the charts are coordinate projections.
    So via a quotient, we can use $F$ to induce $\Phi:\mathbb C/\Lambda\to X$.
    It remains to show that $\Phi$ is a conformal equivalence.
    As it is analytic, it suffices to show that it is bijective.
    It is surjective as it is nonconstant and everything is compact.
    To see it is injective, consider the period parallelogram centered at $0$, that is the parallelogram with vertices $(\omega_1+\omega_2)/2,(\omega_2-\omega_1)/2,(-\omega_1-\omega_2)/2,(\omega_1-\omega_2)/2$.
    It suffices to show that $F$ is injective in the interior of $P$, which will imply injectivity of $\Phi$ in general due to the valency theorem.
    Suppose $F(z)=F(z')$ for $z,z'\in P^\circ$, then $\wp(z)=\wp(z')$, so $z=\pm z'$ as $\wp$ is even and has degree $2$.
    But also $\wp^\prime(z)=\wp^\prime(z')=\pm \wp^\prime(z)$ as $\wp$ is odd, so $z=z'$ as the zeros of $\wp^\prime$ are not in $P^\circ$.
    This completes the proof.
\end{proof}