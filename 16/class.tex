\subsection{Classification of Riemann Surfaces}
We roughly classified all Riemann surfaces in the preceding section by viewing them as quotients of their universal covers, which can only be one of $\mathbb C_\infty,\mathbb C,\mathbb D$.
We say the surface is uniformised by its universal cover.
For some of these cases, we can do something better.
\begin{proposition}
    If a Riemann surface $R$ is uniformised by $\mathbb C_\infty$, then $R$ is conformally equivalent to $\mathbb C_\infty$.
\end{proposition}
\begin{proof}
    Suppose $R=G\backslash\mathbb C_\infty$, then we know that $G$ acts by conformal equivalences $\mathbb C_\infty\to\mathbb C_\infty$.
    But this is just M\"obius transformations (from Example Sheet).
    However, any M\"obius transformation has at least one fixed point, but $G$ should act freely, hence necessarily $G$ is trivial and hence $R\cong\mathbb C_\infty$.
\end{proof}
What about $\mathbb C$?
\begin{proposition}
    If a Riemann surface $R$ is uniformised by $\mathbb C$, i.e. $R\cong G\backslash\mathbb C$, then one of the following holds:\\
    (i) $G$ is trivial and $R\cong\mathbb C$.\\
    (ii) $G\cong\mathbb Z$ and $R\cong\mathbb C_\star$.\\
    (iii) $G\cong\mathbb Z^2$ and $R\cong\mathbb C/\Lambda$ for a lattice $\Lambda$.
\end{proposition}
\begin{proof}
    The conformal automorphisms of $\mathbb C$ are simply the (nonconstant) linear maps $\{z\mapsto az+b:a\in\mathbb C_\star,b\in\mathbb C\}$ (again from Example Sheet).
    Which of them can $G$ act by?
    Note that if $a\neq 1$, then $z\mapsto az+b$ has a fixed point, therefore $G$ can only consist of translations.
    But then (Example Sheet again!) $G$ (identified by the values of $b$) can only be one of:\\
    (i) Trivial.\\
    (ii) $\langle\omega\rangle\cong\mathbb Z$ for some $\omega\neq 0$.\\
    (iii) a lattice $\Lambda\cong\mathbb Z^2$.\\
    And these corresponds to the three situations as stated.
\end{proof}
Oh, by the way, a surface cannot be uniformised by more than one of $\mathbb C_\infty,\mathbb C,\mathbb D$.
\begin{lemma}[Lifting Lemma]
    Let $f:R\to S$ be an analytic map of Riemann surfaces.
    Suppose $R$ is simply connected, and let $\pi:\tilde{S}\to S$ be the uniformising map of $S$, then there is an analytic map $F:R\to\tilde{S}$ such that $f=\pi\circ F$.
\end{lemma}
So we have the commutative diagram
\[
    \begin{tikzcd}
        &\tilde{S}\arrow{d}{\pi}\\
        R\arrow[dashed]{ur}{F}\arrow[swap]{r}{f}&S
    \end{tikzcd}
\]
\begin{proof}
    Example Sheet.
\end{proof}
\begin{proposition}
    A Riemann surface $R$ is uniformised by at most one of $\mathbb C_\infty,\mathbb C,\mathbb D$.
\end{proposition}
\begin{proof}
    By the previous discussion, we already know everything that is uniformised by $\mathbb C$ or $\mathbb C_\infty$ and they are distinct.
    Now suppose $R$ is uniformised by $\mathbb D$ and $\tilde{R}$ which is either $\mathbb C$ or $\mathbb C_\infty$.
    Let $\pi,f$ be the respective uniformisation maps, then by the preceding lemma, there is $F:\tilde{R}\to\mathbb D$ such that
    \[
        \begin{tikzcd}
            &\mathbb D\arrow{d}{\pi}\\
            \tilde{R}\arrow[swap]{r}{f}\arrow{ur}{F}&R
        \end{tikzcd}
    \]
    commutes.
    But then $F$ has to be constant by Liouville's Theorem, hence $f$ is also constant, contradiction.
\end{proof}
So any other Riemann surface must be uniformised by $\mathbb D$.
\begin{proposition}
    Any conformal automorphisms of $\mathbb D$ is in the form
    $$z\mapsto e^{i\theta}\frac{z-a}{1-\bar{a}z},a\in\mathbb C,\theta\in\mathbb R$$
\end{proposition}
\begin{proof}
    Complex Analysis.
\end{proof}
This is perhaps easier to picture if we send $\mathbb D$ to the open upper half plane $\mathbb H$ (by the M\"obius transformation $z\mapsto (1+iz)/(1-iz)$), which has automorphisms in the form $z\mapsto (az+b)/(cz+d)$ for $a,b,c,d\in\mathbb R,ad-bc=1$.
\begin{definition}
    A subgroup of $\operatorname{PSL}(\mathbb R)$ that acts properly discontinuously on $\mathbb H$ is called a Fuchsian group.
\end{definition}