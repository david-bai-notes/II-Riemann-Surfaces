\section{Uniformisation and its Consequences}
\subsection{The Uniformisation Theorem}
\begin{theorem}
    Every simply connected Riemann surfaces is conformally equivalent to either $\mathbb C_\infty$, $\mathbb C$ or $\mathbb D=D(0,1)$.
\end{theorem}
\begin{proof}
    Well beyond the scope of this course (duh!).
\end{proof}
Note that despite the existence of conformal equivalence, it might be very difficult to actually find one.
\begin{remark}
    The three items on this list are all conformally distinct.
    $\mathbb C_\infty$ is obviously distinct from the two others since it is compact.
    To see $\mathbb D$ and $\mathbb C$ are conformally distinct, just observe that there is no nonconstant analytic map $\mathbb C\to\mathbb D$ by Liouville's Theorem yet there is the (nonconstant) analytic natural inclusion $\mathbb C\hookrightarrow\mathbb C_\infty$.
\end{remark}
The uniformisation is extremely useful as it links together the topological properties and possible analytic structures of certain topological surfaces.
\begin{corollary}
    Any conformal structure on $S^2$ is conformally equivalent to $\mathbb C_\infty$.
\end{corollary}
\begin{proof}
    $S^2$ is compact and simply connected.
\end{proof}
How about surfaces with positive genus?
They are not simply connected, but we know that it has a nice universal cover (Algebraic Topology again!).
\begin{theorem}
    Every Riemann surface $R$ has a regular covering map $\pi:\tilde{R}\to R$ such that $\tilde{R}$ is simply connected.
    Furthermore, there is a group $G$ acting freely and properly discontinuously by conformal equivalences on $\tilde{R}$ and the covering map descends to a conformal equivalence $G\backslash\tilde{R}\cong R$.
\end{theorem}
\begin{proof}
    Algebraic Topology.
\end{proof}
Therefore
\begin{corollary}
    Every Riemann surface $R$ is conformally equivalent to a quotient $R\cong G\backslash\tilde{R}$ where $\tilde{R}$ is conformally equivalent to one of $\mathbb C_\infty,\mathbb C,\mathbb D$, and $G$ acts freely and properly discontinuously
\end{corollary}
\begin{proof}
    Immediate.
\end{proof}
\begin{remark}
    In fact, $G$ is just $\pi_1(R)$ acting by deck transformations.
    Or in this context, $G$ is the collection of conformal equivalences $\phi:\tilde{R}\to\tilde{R}$ such that $\pi\circ\phi=\pi$.
\end{remark}