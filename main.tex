\documentclass[a4paper]{article}

\usepackage{hyperref}

\newcommand{\triposcourse}{Riemann Surfaces}
\newcommand{\triposterm}{Michaelmas 2020}
\newcommand{\triposlecturer}{Dr. H. Wilton}
\newcommand{\tripospart}{II}

\usepackage{amsmath}
\usepackage{amssymb}
\usepackage{amsthm}
\usepackage{mathrsfs}

\usepackage{tikz-cd}

\theoremstyle{plain}
\newtheorem{theorem}{Theorem}[section]
\newtheorem{lemma}[theorem]{Lemma}
\newtheorem{proposition}[theorem]{Proposition}
\newtheorem{corollary}[theorem]{Corollary}
\newtheorem{problem}[theorem]{Problem}
\newtheorem*{claim}{Claim}

\theoremstyle{definition}
\newtheorem{definition}{Definition}[section]
\newtheorem{conjecture}{Conjecture}[section]
\newtheorem{example}{Example}[section]

\theoremstyle{remark}
\newtheorem*{remark}{Remark}
\newtheorem*{note}{Note}

\title{\triposcourse{}
\thanks{Based on the lectures under the same name taught by \triposlecturer{} in \triposterm{}.}}
\author{Zhiyuan Bai}
\date{Compiled on \today}

%\setcounter{section}{-1}

\begin{document}
    \maketitle
    This document serves as a set of revision materials for the Cambridge Mathematical Tripos Part \tripospart{} course \textit{\triposcourse{}} in \triposterm{}.
    However, despite its primary focus, readers should note that it is NOT a verbatim recall of the lectures, since the author might have made further amendments in the content.
    Therefore, there should always be provisions for errors and typos while this material is being used.
    \tableofcontents
    \section{Analytic and Meromorphic Functions}
We have seen previously that there can be some multivalued functions in $\mathbb C$ carries important meanings.
Most important examples being $\log$ and $\sqrt[m]{\cdot}$.
What do we really mean by these multivalued functions?
When we evaluate them, we always have to choose a branch cut, but it does not capture the relationship between different branch cuts.
We shall try to find a way to characterise these functions in a proper way that realises their properties as they deserve.
\subsection{Analytic Functions and their Zeros}
\begin{definition}
    A domain is an open, connected subset $D\subset\mathbb C$.
\end{definition}
\begin{example}
    Open disks, annuli, punctured disks are domains.
\end{example}
\begin{definition}
    Let $D\subset\mathbb C$ be a domain.
    A function $f:D\to\mathbb C$ is holomorphic or analytic if either it is $\mathbb C$-differentiable everywhere on $D$ or it has a local Taylor series around every point in $D$.
\end{definition}
We know the two criteria above are equivalent from discussions back in IB Complex Analysis.
\begin{proposition}
    Let $f:D\to\mathbb C$ be an analytic functions on a domain.
    If $f(z_0)=0$, then either $f$ is identically zero or nowhere zero on a punctured disk centering at $z_0$.
\end{proposition}
\begin{proof}
    Obvious and already discussed in Complex Analysis, but let's do it again.
    Consider the Taylor series of $f$ in a neighbourhood $U$ of $z_0$.
    So for $z\in U$ we have
    $$f(z)=\sum_{n=0}^\infty a_n(z-z_0)^n$$
    If $f$ is not identically zero, then we can choose minimal $m$ such that $a_m\neq 0$, so $f(z)=(z-z_0)^mg(z)$ where
    $$g(z)=\sum_{n=0}^\infty a_{m+n}(z-z_0)^n\neq 0$$
    is nonzero at $z_0$, hence is nowhere zero in a punctured disk around $z_0$ in $U\subset D$ by continuity.
    $f$ is then nowhere zero in the same punctured disk.
\end{proof}
\begin{corollary}[Identity Principle]
    Let $f,g$ be analytic functions defined on a domain $D\subset\mathbb C$.
    If the subspace $\{z\in D|f(z)=g(z)\}$ is not discrete, then $f\equiv g$ on $D$.
\end{corollary}
\begin{proof}
    Immediate.
\end{proof}
\subsection{Meromorphic Functions and Singularities}
\begin{definition}
    A function $f$ defined on a punctured disk around $z_0$ is said to have a isolated singularity at $z_0$.
\end{definition}
\begin{proposition}
    If an analysic function $f$ has an isolated singularity at $z_0$, then $f$ has a Laurent series
    $$f(z)=\sum_{n=-\infty}^\infty a_n(z-z_0)^n$$
    on a punctured disk around $z_0$.
\end{proposition}
\begin{proof}
    Complex Analysis.
\end{proof}
\begin{definition}
    If $a_n=0$ for $n<0$, then $z_0$ is said to be a removable singularity of $f$.\\
    If $a_n=0$ for $n<-m<0$ and $a_{-m}\neq 0$, then we say $z_0$ is a pole of order $m$.\\
    Otherwise, we say $z_0$ is an essential singularity.
\end{definition}
\begin{theorem}
    If $f$ is bounded near $z_0$, then $z_0$ has to be a removable singularity.
\end{theorem}
\begin{proof}
    Complex Analysis.
\end{proof}
\begin{theorem}[Casorati-Weierstrass]
    $z_0$ is an essential singularity iff $f(U)$ is dense in $\mathbb C$ for any punctured neighbourhood of $z_0$.
\end{theorem}
\begin{proof}
    Complex Analysis.
\end{proof}
\begin{definition}
    If a holomorphic function $f:D\setminus A\to\mathbb C$ for some domain $D$ and discrete $A$ has poles at the points of $A$, then $f$ is said to be meromorphic.
\end{definition}
\begin{example}
    The function $f(z)=1/(e^{1/z}-1)$ is meromorphic where one takes $D$ to be the open upper half-plane and $A=\{1/(2\pi in):n\in\mathbb N\}$.
    In particular, the poles are all simple (i.e. of order $1$).\\
    Note that the function can be extended to the whole of $\mathbb C$ except at the discrete set $\{1/(2\pi in):n\in\mathbb Z\setminus\{0\}\}\cup\{0\}$, which contains $0$ as an essential singularity and others as simple poles.
\end{example}
\subsection{Analytic Continuation}
\begin{definition}
    A function element $F=(f,U)$ on a domain $D$ consists of a subdomain $U\subset D$ and an analytic function $f:U\to\mathbb C$.
\end{definition}
\begin{lemma}[Direct Analytic Continuation]
    Let $(f,U),(g,V)$ are function elements such that $U\cap V\neq\varnothing$ and $f=g$ on $U\cap V$, then $f$ determines $g$.
\end{lemma}
\begin{proof}
    Identity Principle.
\end{proof}
We write $(f,U)\sim (g,V)$ for direct analytic continuation.
\begin{definition}
    Analytic continuation is a sequence of iteration of direct analytic continuations that get from one function element to another.
    We write $(f,U)\approx (g,V)$ in this case.
\end{definition}
\begin{remark}
    $\approx$ is an equivalence relation.
\end{remark}
\begin{definition}
    A $\approx$-equivalence class $\mathcal F$ of function elements on a domain $D$ is called a complete analytic function on $D$.
\end{definition}
\subsection{The Complex Logarithm}
Let $\mathbb C_\star=\mathbb C\setminus\{0\}$.
We want to invert the exponential function $\exp:\mathbb C\to\mathbb C_\star$, but $\exp$ is not injective on $\mathbb C$.
We used to see $\log$ as a multivalued function to deal with this problem, but in fact, we can see it as a complete analytic function on $\mathbb C_\star$.\\
Indeed, for $(\alpha,\beta)\subset\mathbb R$ with $|\alpha-\beta|<2\pi$, we can define
$$U_{(\alpha,\beta)}=\{re^{i\theta}|r>0,\alpha<\theta<\beta\},f_{(\alpha,\beta)}(z)=\log r+i\theta,r=|z|,\theta\in (\alpha,\beta)$$
Then $F_{(\alpha,\beta)}=(f_{(\alpha,\beta)},U_{(\alpha,\beta)})$ is a collection of function elements.
Let $I(n)=((n-1)\pi/2,(n+1)\pi/2)$ for $n\in\mathbb Z$.
\begin{proposition}
    $F_{I(n)}\sim F_{I(m)}$ iff $|m-n|\le 1$.
\end{proposition}
\begin{proof}
    Just do a case analysis based on $m-n\bmod 4$.
\end{proof}
\begin{corollary}
    $F_{I(m)}\approx F_{I(n)}$ for any $m,n\in\mathbb Z$.
\end{corollary}
\begin{proof}
    Immediate.
\end{proof}
So this characterises a complete analytic function that is the complex logarithm.
\begin{remark}
    We know that $f_{I(0)}(1)=0$ but $f_{I(4)}(1)=2\pi i$, so analytic continuation in this way is not unique.
    But we have ``pasted'' them together to make it a complete analytic function.
\end{remark}
\begin{definition}
    Let $\gamma:[0,1]\to D$ be a path.
    We say $(f,U)\approx_\gamma (g,V)$ if there is some $0=t_0<t_1<\cdots<t_n=1$ and function elements $(f=f_0,U=U_0),(f_1,U_1),\ldots,(g=f_{n+1},V=U_{n+1})$ such that $(f_i,U_i)\sim (f_{i+1},U_{i+1})$ for $i=0,\ldots,n$ and $\gamma(t_i)\in U_i\cap U_{i+1}$.
    This is called analytic continuation along a path.
\end{definition}
Analytic continuation along a path has the desired uniqueness property.
This fact is known as the Classical Monodromy Theorem.
    \section{Natural Boundary; A Gluing Construction; Roots}
\subsection{Natural Boundary}
Sometimes, it is impossible to do analytic continuation.
Write $\mathbb D=D(0,1)$ and $\mathbb T=\partial\mathbb D=S^1$.
Consider a power series $f(z)=\sum_na_nz^n$ with radius of convergence $1$.\\
We say $z_0\in\mathbb T$ is regular if there is a neighbourhood $U$ of $z_0$ such that there exists a function element $(g,U)$ such that $f|_{U\cap\mathbb D}=g|_{U\cap\mathbb D}$.
A point that is not regular is singular.
Easily, the set of regular points is open in $\mathbb T$, so the set of singular points has to be closed.
There are something to beware of which are illustrated in the example below.
\begin{example}
    Consider $f(z)=\frac{1}{1-z}=\sum_{n\ge 0}z^n$.
    We know that the set of singular point is just $\{1\}$.
    However, the power series evaluated at $-1$ does not converge.
    So a regular point needs not guarantee that the power series converge there.
\end{example}
\begin{example}
    Consider the power series
    $$g(z)=\sum_{n\ge 2}\frac{z^n}{n(n-1)}$$
    Then the series converges at $z=1$, but $1$ is not regular for $g$ because it is singular for $f=g^{\prime\prime}$.
\end{example}
The moral od the examples is whether or not a point is regular does not relate directly to whether or not the power series converges there.
\begin{proposition}
    If a power series $f(z)=\sum_{n\ge 0}a_nz^n$ has radius of convergence $1$, then some point of $\mathbb T$ is singular.
\end{proposition}
\begin{proof}
    Just use the compactness of $\mathbb T$.
\end{proof}
\begin{definition}
    If every point of $\mathbb T$ is singular, we say $\mathbb T$ is the natural boundary of $f$.
\end{definition}
\begin{remark}
    We can extend this definition to other simple curves in $\mathbb C_\infty$.
\end{remark}
\begin{example}
    Consider the function $f(z)=\sum_{n\ge 0}z^{n!}$.
    We shall show that $\mathbb T$ is its natural boundary.
    It suffices to show any point in the form $\omega=e^{2\pi ip/q}$ is singular, where $p,q\in\mathbb Z,q\neq 0$.
    Indeed, fix such an $\omega$ with corresponding $p,q$, then for $r\in(0,1)$,
    $$f(r\omega)=\sum_{n=0}^{q-1}r^{n!}\omega^{n!}+\sum_{n\ge q}r^{n!}$$
    But the second term definitely converge.
    To see this, we observe that for any $M$,
    $$\lim_{r\to 1}\sum_{n=q}^{M+q}r^{n!}=M+1\implies \sum_{n=q}^{M+q}r^{n!}>M$$
    for $r$ sufficiently close to $1$.
    So the second term converges.
    Yet the first term is bounded, therefore $f(r\omega)\to\infty$ as $r\to1$, so $\omega$ has to be singular at $\omega$.
\end{example}
    \subsection{A Gluing Construction}
In previous parts, we saw that we can make $\log$ is a complete analytic function.
But we are not entirely satisfied, as it is just a bunch of function elements related together, instead of a genuin function.
We shall construct a space as a ``bigger domain'' $R$ at which we can realise $\log$ as a geniune function.\\
The idea is to consider the function elements we defined earlier and glue them together.
We define
$$R=\left( \coprod_{n\in\mathbb Z}U_{I(n)} \middle)\right/\sim$$
where $z_1\in U_{I(m)}$ and $z_2\in U_{I(n)}$ have $z_1\sim z_2$ iff $z_1=z_2$ as elements of $\mathbb C$ and $f_{I(m)}(z_1)=f_{I(n)}(z_2)$.
We give $R$ the quotient topology.
One can imagine $R$ as an ``infinite multi-storey carpark'' that spirals up and down.
\begin{remark}
    Since $F_{I(m)}\approx F_{I(n)}$ for all $m,n\in\mathbb Z$, it follows that $R$ is path-connected by following the sequence of direct analytic continuations.
\end{remark}
Then, with this construction, we can extend all the $f_{I(n)}$ on the $U_{I(n)}$ to a global function $f:R\to\mathbb C$ by $f([z])=f_{I(n)}(z)$ for $z\in U_{I(n)}$.
\begin{proposition}
    $f$ is well-defined.
\end{proposition}
\begin{proof}
    Follows directly from our definition of the equivalence relation $\sim$.
\end{proof}
Similarly, the natural inclusions $U_{I(n)}\hookrightarrow\mathbb C_\star$ can be extended to a global function $\pi:R\to\mathbb C_\star$ with $\pi([z])=z$.
One can also easily verify that $\pi$ is well-defined.
There is a very nice relationship between $f,\pi$ and the usual exponential map.
Indeed, $\exp\circ f=\pi$, which basically tells us $f$ has basically everything we want from $\log$.\\
We can use these global functions together.
Define $\Phi([z])=(\pi([z]),f([z]))$, then $\Phi$ is injective by definition of $\sim$.
Therefore $R$ is Hausdorff as $\mathbb C^2$ is.
\begin{remark}
    $\Phi$ indeed identifies $R$ with the graph $\{(w,z)\in\mathbb C^2:w=\exp(z)\}$.
    So we can view $R$ alteratively as ``flipping'' the graph of $\exp$.
\end{remark}
    \subsection{Complex Roots}
Consider the $k^{th}$ power map $p_k:z\mapsto z^k$.
We can ``invert'' it by $\sqrt[k]{z}=\exp(k^{-1}\log z)$.
This multi-valued function can be analysed analogously to how we analyse $\log$.
Take $I(n)=((n-1)\pi/2,(n+1)\pi/2)$ as usual and $U_{I(n)}$ as before equipped with the same $f_{I(n)}:U_{I(n)}\to\mathbb C$ we did for $\log$.
Then consider $g_{I(n)}(z)=\exp(k^{-1}f_{I(n)}(z))$ and the function elements $G_{I(n)}=(g_{I(n)},U_{I(n)})$.
Now $G_{I(n)}$ only depends on $n\bmod k$, so WLOG we can think of $n\in\mathbb Z/k\mathbb Z$.
Everything else is similar to what we did before, and we can get $G_{I(n)}\sim G_{I(m)}$ iff $n-m\equiv 0,\pm 1\pmod{k}$.
Furthermore, a similar gluing constructio defines a path-connected Hausdorff space $R_k$ and maps
\[
    \begin{tikzcd}
        R_k\arrow{r}{g}\arrow[swap]{dr}{\pi}&\mathbb C_\star\arrow{d}{p_k}\\
        &\mathbb C_\star
    \end{tikzcd}
\]
So $g$ gives what we want from the $k^{th}$ root.
    \section{Riemann Surfaces; Analytic Maps}
\subsection{Covering Maps}
In previous sections, we have seen the way of realising the complex logarithm by constructing functions $f,\pi:R\to\mathbb C$ satisfying the $\exp\circ f=\pi$.
That is, the diagram
\[
    \begin{tikzcd}
        R\arrow{r}{f}\arrow[swap]{dr}{\pi}&\mathbb C\arrow{d}{\exp}\\
        &\mathbb C_\star
    \end{tikzcd}
\]
Sadly, $\pi$ is not exactly a homeomorphism, but it is the next best thing.
\begin{definition}
    Let $\tilde{X},X$ be path-connected Hausdorff topological spaces.
    A covering map $\pi:\tilde{X}\to X$ is a local homeomorphism.
    That is, each $\tilde{x}\in\tilde{X}$ has an open neighbourhood $\tilde{U}$ such that $\pi|_{\tilde{U}}$ is a homoemorphism onto its image.
\end{definition}
\begin{definition}
    A covering map $\pi:\tilde{X}\to X$ is regular if for each $x\in X$ there is an open neighbourhood $U$ of $x$ and a discrete set $\Delta_x$ such that $\pi^{-1}(U)\cong U\times\Delta_x$ and the diagram
    \[
        \begin{tikzcd}
            \pi^{-1}(U)\arrow{r}{\cong}\arrow[swap]{dr}{\pi}&U\times\Delta_x\arrow{d}{(u,\delta)\mapsto u}\\
            &U
        \end{tikzcd}
    \]
    commutes.
\end{definition}
In particular, $\pi|_{\pi^{-1}(U)}$ must has image $U$.
A useful and non-confusing way of taking $U\times\Delta_x$ is to think of it as a disjoint union of copies of $U$.
\begin{example}
    1. The map $\pi:R\to\mathbb C$ we defined when treating $\log$ is a regular convering map as
    $$\pi^{-1}(U_{I(n)})=\coprod_{m\equiv n\pmod{4}}U_{I(m)}\cong U_{I(n)}\times\mathbb Z$$
    2. For each open interval $I\subset\mathbb R$, write
    $$\tilde{V}_I=\mathbb R+iI=\{x+iy:x\in\mathbb R,y\in I\}$$
    As long as the length of $I$ is at most $2\pi$, the exponential function restricts to a homeomorphism $\tilde{V}_I\to U_I$ with the obvious inverse obtained by taking a branch of $\log$.
    So
    $$\exp^{-1}(U_{I(n)})=\coprod_{m\equiv n\pmod{4}}\tilde{V}_{I(m)}\cong U_{I(n)}\times\mathbb Z$$
    which means $\exp:\mathbb C\to\mathbb C_\star$ is a covering map.\\
    3. (non-example) Consider $\pi:\mathbb D\to\mathbb C$ which is obviously a covering map and $z\in\mathbb T$ with a neighbourhood $U\ni z$, then $\pi^{-1}(U)=U\cap\mathbb D$.
    But then the image of $\pi|_{U\cap\mathbb D}$ is never $U$, so $\pi$ is not regular.\\
    4. The map $\pi:R_k\to\mathbb C_\star$ we constructed for $\sqrt[k]{\cdot}$ is also regular.
\end{example}
    \subsection{Abstract Riemann Surfaces}
As we have seen previously, when we wanted to treat stuff like $\log$ and $\sqrt[k]{\cdot}$ formally as functions, the most useful way is to obtain a bigger domain that has some topological and analytical characteristics of $\mathbb C$.
We want to do complex analysis on these ``bigger domains'' which seem to have better properties than just $\mathbb C$.
This motivated the study of Riemann surfaces.
\begin{definition}
    Let $R$ be a topological space.
    A chart on $R$ is a pair $(\phi,U)$ where $U$ is an open subset of $R$ and $\phi:U\to D$ is a homeomorphism to an open subset of $\mathbb C$.
    A set of charts $\mathcal A$ is an atlas on $R$ if:\\
    1.
    $$\bigcup_{(\phi,U)\in\mathcal A}U=R$$
    2. For $(\phi_1,U_1),(\phi_2,U_2)\in\mathcal A$ with $U_1\cap U_2\neq\varnothing$,
    $$\phi_1\circ\phi_2^{-1}=(\phi|_{U_1\cap U_2})\circ (\phi_2|_{U_1\cap U_2})^{-1}$$
    is analytic on $\phi_2(U_1\cap U_2)$.
\end{definition}
This composition $\phi_1\circ\phi_2^{-1}$ is called a transition function.
\begin{remark}
    As $(\phi_1\circ\phi_2^{-1})^{-1}=\phi_2\circ\phi_1^{-1}$, the transition function is biholomorphic.
\end{remark}
\begin{example}
    Take $R=\mathbb C$.\\
    1. $\mathcal A$ consisting of an open cover of $R$ paired with the obvious identity maps is an atlas.\\
    2. $\mathcal A'$ consisting of an open cover of $R$ paired with $z\mapsto z+1$ is also an atlas.\\
    3. $\mathcal A\cup\mathcal A'$ is also an atlas!
\end{example}
This example motivated an extended definition that ask the atlas to include as much information as possible about the analytic structure of $R$.
\begin{definition}
    A conformal structure on $R$ is an atlas $\mathcal A$ on $R$ which is maximal in the sense that if a chart $(\psi,V)$ on $R$ has the property that for any $(\phi,U)\in\mathcal A$ with $U\cap V\neq\varnothing$, the transition function $\phi\circ\psi^{-1}$ is analytic, then $(\psi,V)\in\mathcal A$.
\end{definition}
We can define Riemann surface now.
\begin{definition}
    A Riemann surface is a pair $(R,\mathcal A)$ where $R$ is a path-connected Hausdorff topological space and $\mathcal A$ is a conformal structure on $R$.
\end{definition}
By abuse of notation, we often just write $R$ to denote a Riemann surface when the conformal structure it is equipped with is understood.
\begin{lemma}
    Every atlas $\mathcal A$ is contained in a unique conformal structure $\hat{A}$.
\end{lemma}
\begin{proof}
    Make the obvious choice of $\hat{\mathcal A}$ which consists of all charts $(\psi,V)$ on $R$ such that $\psi\circ\phi^{-1}$ is analytic for every $(\phi,U)\in A$.
    This is necessarily maximal.
    To see it is an atlas, take $(\psi_1,V_1),(\psi_2,V_2)\in\hat{\mathcal A}$ and arbitrary $p\in V_1\cap V_2$.
    As $\mathcal A$ is an atlas, there is $(\phi,U)\in\mathcal A$ such that $p\in U$, so
    $$\psi_1\circ\psi_2^{-1}=(\psi_1\circ\phi^{-1})\circ(\phi\circ\psi_2^{-1})$$
    It then follows that $\psi_1\circ\psi_2^{-1}$ is analytic at $\psi_2(p)$.
    As this works for any $p\in V_1\cap V_2$, we get $\psi_1\circ\psi_2^{-1}$ is analytic at $\psi_2(V_1\cap V_2)$, so $\hat{\mathcal A}$ is an atlas, hence a conformal structure.
    Uniqueness is obvious.
\end{proof}
Therefore, we can identify a conformal structure on a Riemann surface just by an atlas defined on it.
Also for any atlas on a Riemann surface, we can extend it to a conformal structure.
\begin{example}\label{c_canon}
    The atlas $\mathcal A$ consisting of an open cover equipped with identity maps extends to a unique conformal structure $\hat{\mathcal A}$ on $\mathbb C$ that makes it a Riemann surface.
\end{example}
But this is not the only conformal structure we can define on $\mathbb C$.
\begin{example}
    The atlas $\bar{\mathcal A}$ consisting of an open cover with the conjugate map is an atlas, but is quite obviously not contained in $\hat{\mathcal A}$ as in above.
    This Riemann surface with this alternative conformal structure is denoted by $\bar{\mathbb C}$.
\end{example}
\begin{definition}
    The conformal structure in Example \ref{c_canon} is called the canonical conformal structure on $\mathbb C$.
\end{definition}
When we simply talk about $\mathbb C$ as a Riemann surface, we shall always refer to the one equipped with the canonical conformal structure.
\begin{example}
    If $R$ is a Riemann surface with conformal structure $\mathcal A$ and $S\subset R$ is open, then
    $$\{(\phi|_{S\cap U},U\cap S)|(\phi,U)\in\mathcal A\}$$
    is an atlas on $S$.
    In particular, any domain $D\subset\mathbb C$ (like $\mathbb C_\star$) can be made a Riemann surface in this way.
\end{example}
\begin{example}[Riemann Sphere]
    By stereographic projection, we have $\mathbb C_\infty=\mathbb C\cup\{\infty\}\cong S^2$.
    Let $\mathcal A=\{(\operatorname{id},\mathbb C),(\phi,U)\}$ where $U=\mathbb C_\star\cup\{\infty\}$ with $\phi(z)=1/z$.
    The transition function is $z\mapsto 1/z$ on $\mathbb C_\star$ which is analytic, so $\mathcal A$ is indeed an atlas.
\end{example}
    \subsection{Analytic Maps}
\begin{definition}
    Let $R$ and $S$ be Riemann surfaces.
    A continuous map $f:R\to S$ is analytic or holomorphic if for any chart $(\phi,U)$ on $R$ and $(\psi,V)$ on $S$, the map $\psi\circ f\circ \phi^{-1}$ is analytic on $\phi(U\cap f^{-1}(V))$.
\end{definition}
\begin{lemma}\label{analytic_local}
    A continuous map $f:R\to S$ of Riemann surfaces is analytic iff for each $p\in R$, there is a chart $\phi_p,U_p$ on $R$ with $p\in U_p$ and a chart $(\psi_p,V_p)$ on $V$ with $f(p)\in V_p$ such that $\psi_p\circ f\circ \phi_p^{-1}$ is analytic on $\phi_p(U_p\cap f^{-1}(V_p))$
\end{lemma}
The key point of the proof is basically that the function we want to be analytic can be written as a composition of local analytic functions composed with transition maps.
\begin{proof}
    The ``only if'' direction is immediate.
    For the ``if'' direction, given charts $(\phi,U)$ on $R$ and $(\psi,V)$ on $S$, it suffices to show that $\psi\circ f\circ\phi^{-1}$ is analytic at $\phi(p)$ for any $p\in U\cap f^{-1}(V)$.
    We know from hypothesis that there is some charts $(\phi_p,U_p)$ on $R$ and $(\psi_p,V_p)$ on $S$ with $p\in U_p,f(p)\in V_p$ and $\psi_p\circ f\circ\phi_p^{-1}$ is analytic on $\phi_p(U\cap f^{-1}(V))$, hence in particular at $\phi_p(p)$.
    Hence
    $$\psi\circ f\circ\phi^{-1}=(\psi\circ\psi_p^{-1})\circ(\psi_p\circ f\circ\phi_p^{-1})\circ (\phi_p\circ\phi^{-1})$$
    is analytic at $\phi(p)$.
\end{proof}
\begin{lemma}\label{analytic_compose}
    If $f:R\to S$ and $g:S\to T$ are analytic, so is $g\circ f:R\to T$.
\end{lemma}
\begin{proof}
    Simple corollary of Lemma \ref{analytic_local}.
\end{proof}
\begin{definition}
    A conformal equivalence or biholomorphism is an analytic bijection of Riemann surfaces with analytic inverse.
\end{definition}
It is an equivalence relation by Lemma \ref{analytic_compose}.
\begin{example}
    The map $f:\mathbb C\to\bar{\mathbb C}$ with $z\mapsto \bar{z}$ is a conformal equivalence.
\end{example}
    \section{Examples of Conformal Structures; Analytic Functions}
\subsection{Covering Maps and Analyticity}
\begin{lemma}\label{covering_conformal}
    If $\pi:\tilde{R}\to R$ is a covering map where $R$ is a Riemann surface, then there is a unique conformal structure on $\tilde{R}$ such that $\pi$ is analytic.
\end{lemma}
\begin{proof}
    We construct an atlas on $\tilde{R}$ as follows:
    For $p\in\tilde{R}$, as $\pi$ is a covering map, there is an open neighbourhood $\tilde{N}_p$ of $p$ such that $\pi|_{\tilde{N}_p}:\tilde{N}_p\to N_p$ is a homeomorphism for some neighbourhood $N(p)$ of $\pi(p)$.
    Now there is a conformal structure on $R$, so we can pick a chart $(\phi_p,U_p)$ on $R$ such that $p\in U_p$.
    Our desired chart is then $(\tilde{\phi}_p,\tilde{U}_p)$ where $\tilde{U}_p=(\pi|_{\tilde{N}_p})^{-1}(N_p\cap U_p)$ and $\tilde{\phi}_p=\phi_p|_{N_p\cap U_p}\circ\pi|_{\tilde{U}_p}$.
    It is easy to see it is a chart.
    The collection of all charts produced in this form is an atlas.
    To see this, just observe that the transition function is
    $$\tilde{\phi}_p\circ\tilde{\phi}_q^{-1}=\phi_p\circ\pi\circ\pi^{-1}\circ\phi_q^{-1}=\phi_p\circ\phi_q^{-1}$$
    (with proper restrictions to local open sets) which is analytic as we got $\phi_p,\phi_q$ from another atlas.
    So it extends to a conformal structure $\tilde{A}$ on $\tilde{R}$.\\
    We shall show that $\pi$ is analytic in this choice of conformal structure.
    Let $p\in\tilde{R}$ and choose chart $(\phi_p,U_p)$ in $R$ around $\pi(p)$ and chart $(\tilde{\phi}_p,\tilde{U}_p)$ in $\tilde{R}$ constructed from $(\phi_p,U_p)$ in the above way.
    Then with proper restrictions,
    $$\phi_p\circ\pi\circ\tilde{\phi}_p^{-1}=\phi_p\circ\pi\circ\pi^{-1}\circ\phi_p^{-1}=\operatorname{id}_{\mathbb C}$$
    which is analytic.
    We are then done by Lemma \ref{analytic_local}.\\
    To see the uniqueness of this conformal structure, let $\tilde{B}$ be any conformal structure on $\tilde{R}$ such that $\pi$ is analytic.
    Let $p\in\tilde{R}$, $(\psi,V)\in\tilde{B}$ a chart around $p$ and $(\phi_p,U_p)$ a chart around $f(p)$.
    But then locally the transition function $\tilde\phi_p\circ\psi^{-1}=\phi_p\circ\pi\circ\psi^{-1}$ is analytic.
    So by maximality $\tilde{B}=\tilde{A}$.
\end{proof}
\begin{example}
    Consider the Riemann surface $R$ associated with $\log$ we considered earlier.
    Let $f,\pi$ be the functions as usual.
    As $\pi$ is a covering map, there is a unique way to make $R$ a Riemann surface by the preceding lemma where $\pi:R\to C_\star$ is analytic.
    Furthermore, locally $f|_{U_{I(n)}}=f_{I(n)}\circ\pi$, so $f$ is analytic as well.\\
    Furthermore, we know that there is a homeomorphism $f_{I(n)}:U_{I(n)}\to\tilde{V}_{I(n)}$ (where $\tilde{V}_I=\mathbb R+iI$) having $\exp|_{\tilde{V}_{I(U)}}$ as inverse.
    Then $f_{I(n)}^{-1}$ agree wherever their domains intersect, so we can piece them together to give a conformal equivalence between $R$ and $\mathbb C$.
\end{example}
\begin{example}
    Similar case happened with $R_k$ and $\sqrt[k]{\cdot}$.
    Again $\pi$ induces a unique conformal structure on $R_k$ that makes it and $g$ analytic.
    By the same argument as above, we get $g$ is a conformal equivalence.\\
    Actually, we can even do better with this example.
    Note that the singularities in $0$ and $\infty$ are removable by identifying $\hat{p}_k(0)=\hat{\pi}(0)=\hat{g}(0)=0$ and $\hat{p}_k(\infty)=\hat{\pi}(\infty)=\hat{g}(\infty)=\infty$ (where $\hat{g},\hat{p}_k,\hat{\pi}$ are our notation for $g,p_k,\pi$ with this extended domain and codomain).
    This gives
    \[
        \begin{tikzcd}
            \hat{R}_k=R_k\cup\{0,\infty\}\arrow{r}{\hat{g}}\arrow[swap]{dr}{\hat{\pi}}&\mathbb C_\infty\arrow{d}{\hat{p}_k}\\
            &\mathbb C_\infty
        \end{tikzcd}
    \]
    which is pretty nice except now $\hat{\pi}$ is not a covering map anymore.
\end{example}
    \subsection{Analytic Functions}
\begin{definition}
    An analytic function on a Riemann surface $R$ is an analytic map $R\to\mathbb C$.
\end{definition}
We can put analytic functions into a nice form by our study of this structure of Riemann surfaces.
\begin{theorem}[Inverse Function Theorem]
    Let $f$ be an analytic function on a domain $S\subset\mathbb C$.
    If $f^\prime(z_0)\neq 0$ for $z_0\in D$, then there are open neighbourhoods $U$ of $z_0$ and $V$ of $f(z_0)$ such that $f$ restricts to a biholomorphism $U\to V$.
\end{theorem}
\begin{proof}
    Omitted.
\end{proof}
\begin{proposition}\label{local_p_k}
    Let $f$ be a non-constant analytic function on a Riemann surface $R$ and $p\in R$ be a zero of $f$.
    There is a chart $(\phi,U)$ about $p$ with $\phi(p)=0$ such that $f\circ\phi^{-1}(z)=z^m$ for some integer $m>0$.
\end{proposition}
\begin{proof}
    Let $(\psi,V)$ be a chart with $\psi(p)=0$ as adding a constant does not change anything.
    $f$ is not globally constant, so it is not locally constant by the identity principle for Riemann surfaces.
    \footnote{Proved in example sheet.}
    Therefore is an analytic $g$ in a neighbourhood $W\subset\psi(V)$ of $0$ such that $f\circ\psi^{-1}(z)=z^mg(z)$ with $g(0)\neq 0$.\\
    But then since $g$ is continuous, there is $\delta>0$ such that $D(0,\delta)\subset W$ and $g(D(0,\delta))\subset D(g(0),|g(0)|)$ does not contain $0$.
    So there is an analytic branch cut of $\sqrt[m]{\cdot}$ on $g(D(0,\delta))$.
    Define $h(z)=z\cdot\sqrt[m]{g(z)}$ on $D(0,\delta)$, then $f\circ\psi^{-1}(z)=(h(z))^n$.
    Differentiating $h$ gives $h^\prime(0)=\sqrt[m]{g(0)}\neq 0$, so $h$ has an analytic inverse on $D(0,\epsilon)$ for some $0<\epsilon\le\delta$.
    Then $\phi=h\circ\psi$ and $U=\phi^{-1}(D(0,\epsilon))$ gives the required chart as
    $$f\circ\phi^{-1}(z)=f\circ\psi^{-1}\circ h^{-1}(z)=(h(h^{-1}(z)))^m=z^m$$
    which is what we wanted.
\end{proof}
    \section{Complex Tori; the Open Mapping Theorem}
\subsection{Complex Tori}
So far, we only know one compact Riemann surface, namely the Riemann sphere $\mathbb C_\infty$.
One can also picture a torus being compact and can admit a conformal structure.
We shall formalise such constructions.\\
Let $\tau_1,\tau_2\in\mathbb C_\star$ such that they are linearly independent over $\mathbb R$.
Let $\Lambda$ be the additive subgroup generated by $\tau_1,\tau_2$ which is a lattice.
We then define the torus as the quotient group $T=\mathbb C/\Lambda$ that inherits the quotient topology.
We can study this topology via the fundamental parallelogram $P$ with vertices $0,\tau_1,\tau_2,\tau_1+\tau_2$.
With a little geometrical intuition we are hinted that $T\cong S^1\times S^1$ since we are just gluing the sides of the fundamental parallelogram as topological spaces which is easy enough to check.
Consequently it is compact.\\
Now the quotient map $\pi:\mathbb C\to T=\mathbb C/\Lambda$ is a regular covering map.
To prove this, take $0<\epsilon<\min\{|\lambda|:\lambda\in\Lambda\setminus\{0\}\}/2$ which one can verify is well-defined.
Then
$$\pi^{-1}(\pi(D(z_0,\epsilon)))=\bigcup_{\lambda\in\Lambda}(D(z_0,\epsilon)+\lambda)=\coprod_{\lambda\in\Lambda}(D(z_0,\epsilon)+\lambda)\cong D(z_0,\epsilon)\times\Lambda$$
by our definition of $\epsilon$.
Here $\Lambda$ has the discrete topology it inherited as a subspace of $\mathbb C$.
So $\pi$ is indeed a regular covering map.
Now, we use $\pi$ to construct an atlas on $T$.
For $p=z_0+\Lambda\in T$, let $U=\pi(D(z_0,\epsilon))$ for $\epsilon>0$ as before and $(\phi,U)$ is a chart where $\phi=(\pi|_{D(z_0,\epsilon)})^{-1}$.
This works since $\pi$ is a regular covering map.
Now for any other chart constructed in this way, i.e. $(\psi,V)=((\pi|_{D(z_1,\epsilon)})^{-1},\pi(D(z_1,\epsilon)))$, then $U\cap V$ is nonempty iff there is some $\lambda\in\Lambda$ (necessarily unique because of the bound on $\epsilon$) such that $|z_0-(z_1+\lambda)|<2\epsilon$.
In this case, the transition function is just $z\mapsto z+\lambda$ which is analytic.
So this is indeed an atlas.
This extends to a conformal structure on $T$ that makes it a Riemann surface.
It is easy to see that all of these tori are homeomorphic as they are all homeomorphic to $S^1\times S^1$.
But (as will be proven in example sheets) there are infinitely many conformal equivalence classes among these tori.
    \subsection{The Open Mapping Theorem}
We want to study complex analytic functions $f:R\to\mathbb C$ where $R$ is a Riemann surface.
The case where $R$ is compact is especially interesting as we can greatly constrain them using the open mapping theorem.
\begin{theorem}\label{open_mapping}
    Any non-constant, analytic map of Riemann surfaces $f:R\to S$ is an open map.
\end{theorem}
\begin{proof}
    By the identity principle for Riemann surfaces shows that $f$ is not constant in any open subset.
    Let $W\subset R$ be open and $p\in W$.
    Pick charts $(\phi,U)$ containing $p$ and $(\psi,V)$ containing $f(p)$, then by the open mapping theorem on complex plane, $\psi\circ f(U\cap W\cap f^{-1}(V))$ is an open neighbourhood of $\psi\circ f(p)$ in $\psi(f(W)\cap V)$, so $f(U\cap W\cap f^{-1}(V))$ is an open neighbourhood of $f(p)$ in $f(W)$.
\end{proof}
\begin{corollary}
    Let $f:R\to S$ be a non-constant, analytic map of Riemann surfaces.
    If $R$ is compact, then $f$ is surjective and $S$ is also compact.
\end{corollary}
\begin{proof}
    By Theorem \ref{open_mapping}, $f(R)$ is open.
    But $R$ is compact, so $f(R)$ is compact, hence $f(R)$ is also closed as $S$ is Hausdorff.
    But then $S$ is path-connected hence connected, therefore $S=f(R)$ and hence is compact.
\end{proof}
\begin{corollary}
    Every analytic function on a compact Riemann surface is constant.
\end{corollary}
\begin{proof}
    $\mathbb C$ is not compact.
\end{proof}
    \subsection{Harmonic Functions}
By the open mapping theorem, a non-constant function $u:D\to\mathbb R$ where $D$ is a domain cannot be analytic.
However, it can be harmonic
\begin{definition}
    Let $D\subset\mathbb C$ be a domain.
    A smooth function $u:D\to\mathbb R$ is harmonic if
    $$\nabla^2 u=\frac{\partial^2u}{\partial x^2}+\frac{\partial^2u}{\partial y^2}=0$$
\end{definition}
\begin{lemma}\label{harmonic_disk}
    Consider a disk $D\subset\mathbb C$, a function $u:D\to\mathbb R$ is harmonic iff $u=\operatorname{Re}(f)$ for an analytic $f$ on $D$.
\end{lemma}
\begin{proof}
    The ``if'' direction is trivial by the Cauchy-Riemann Equations.
    The ``only if'' direction is exercise.
\end{proof}
\begin{definition}
    Let $R$ be a Riemann surface.
    A function $u\to\mathbb R$ is harmonic if for any chart $(\phi,U)$ on $R$ the composition
    $$u\circ\phi^{-1}:U\to\mathbb R$$
    is harmonic.
\end{definition}
\begin{lemma}
    A real function $u$ on $R$ is harmonic iff for any $p\in R$ there exists one chart $(\phi,U)$ on $R$ such that $u\circ\phi^{-1}$ is harmonic.
\end{lemma}
\begin{proof}
    The ``only if'' direction is trivial.
    For the ``if'' direction, we know from Lemma \ref{harmonic_disk} that for any $p\in R$, there is a chart $(\phi,U)$ with $p\in U$ such that $u\circ\phi^{-1}=\operatorname{Re}f$ for some analytic $f$ on a disk contained in $\phi(U)$.
    Hence for any chart $(\psi,V)$ and any $p\in V$, let $(\phi,U)$ be as above,
    $$u\circ\psi^{-1}=(u\circ\phi^{-1})\circ(\phi\circ\psi^{-1})=\operatorname{Re}(f\circ(\phi\circ\psi^{-1}))$$
    is harmonic near $p$.
    Hence $u$ is harmonic.
\end{proof}
\begin{proposition}[Identity Principle for Harmonic Functions]
    Let $u,v$ be harmonic functions on a Riemann surface $R$.
    Then the set $\{p\in R:u(p)=v(p)\}$ is either $R$ or discrete.
\end{proposition}
\begin{proof}
    Exercise.
\end{proof}
\begin{theorem}[Open Mapping Theorem for Harmonic Functions]
    Any non-constant harmonic function $u$ on a Riemann surface $R$ is an open map.
\end{theorem}
\begin{proof}
    Let $W\subset R$ be open and $p\in W$.
    For small enough $U\ni p$ that is contained in $W$ there is a chart $\phi:U\to\mathbb C$ and analytic $f$ such that $u\circ\phi^{-1}=\operatorname{Re}f$.
    The theorem then follows from the identity principle and open mapping theorem for analytic functions.
\end{proof}
\begin{corollary}
    If $R$ is a compact Riemann surface, all harmonic functions on $R$ are constant.
\end{corollary}

    \section{Meromorphic Functions; A Worked Example}
\subsection{Meromorphic Functions}
Recall that the only analytic maps from a compact Riemann surface to $\mathbb C$ are constants since $\mathbb C$ is not compact.
Therefore, it is more useful to consider a bigger and compact range, i.e. the Riemann sphere $\mathbb C_\infty$.
\begin{definition}
    A meromorphic function on a Riemann surface $R$ is an analytic map $f:R\to\mathbb C_\infty$ that is not constantly $\infty$.
\end{definition}
First we do some ground work for consistency.
\begin{proposition}
    Let $D\subset\mathbb C$ be a domain and $A\setminus D$ a discrete subset.
    A function $f:D\to\mathbb C_\infty$ is meromorphic if and only if $f:D\setminus A\to\mathbb C$ is analytic, $f(A)=\{\infty\}$ and $f$ has a pole at each $a\in A$.
\end{proposition}
\begin{proof}
    Quite obvious honestly, but we are gonna go though it.
    For the ``only if'' direction, let $A=f^{-1}(\{\infty\})$, then $f$ obviously restricts analytically to $\mathbb C$ on $D\setminus A$.
    It remains to show that each $a\in A$ is a pole.
    Working in the standard atlas on $\mathbb C_\infty$ and pick that chart about $\infty$.
    We see that $(f(z))^{-1}=(z-a)^mg(z)$ in a neighbourhood of $a$ where $m\ge 1$ and $g$ is analytic with $g(a)\neq 0$.
    Then $f(z)=(z-a)^{-m}/g(z)$ on an even small domain where $g$ does not have any root.
    This shows immediately that $a$ is a pole.\\
    For the ``if'' direction, then again $f$ is obviously analytic on $D\setminus A$.
    At each $a\in A$, we know that $f(z)=(z-a)^{-m}h(z)$ on a neighbourhood of $a$ with $h(0)\neq 0$.
    Restricting to an even smaller domain where $h$ does not have any root, then $1/f(z)=(z-a)^m/h(z)$, so $f$ is analytic at each $a\in A$ as well by checking the standard atlas.
\end{proof}
    \subsection{A Worked Example}
We have already constructed Riemann surfaces associated to the multi-valued functions $\log$ and $\sqrt[k]{\cdot}$.
Here we will treat another function in this way, namely $w=\sqrt{z^3-z}$.
One possible approach to do it is to give a conformal structure on the graph $\{(w,z)\in\mathbb C^2:w^2=z^3-z\}$.
This approach will not be discussed here since we are going to do it on example sheet.\\
We will use our old idea of ``gluing'' again.
As with $\log$, we need to consider some function elements.
Start with $f(z)=z^3-z=z(z+1)(z-1)$.
We know that $\sqrt{\cdot}$ has branches locally near any point but $0$ which corresponds to the roots $0,\pm 1$ of $f$.\\
Let $D=\mathbb C\setminus([-1,0]\cup[1,\infty))$.
We want to construct function elements $g$ on $D$ such that $g(z)^2=f(z)$.
Fix any $z_0\in D$ and let $g(z_0)=w_0$ for a chosen $w_0^2=f(z_0)$.
Now set
$$g(x)=g(z_0)\exp\left( \frac{1}{2}\int_\gamma\frac{f^\prime(\zeta)}{f(\zeta)}\,\mathrm d\zeta \right)$$
for any choice $\gamma$ of path from $z_0$ to $z$.
To see the choice of the path $\gamma$ does not intervene the value of $g$, consider a loop $\gamma$.
By the argument principle,
$$\frac{1}{2\pi i}\int_\gamma\frac{f^\prime(\zeta)}{f(\zeta)}\,\mathrm d\zeta=\sum_{z_i\text{ zeros of }f}n(\gamma,z_i)-\sum_{p_j\text{ poles of }f}n(\gamma,p_j)$$
where $n$ denote the winding number.
In this particular case, this evaluates to $n(\gamma,0)+n(\gamma,-1)+n(\gamma,1)$.
In example sheet, we have shown that $n(\gamma,1)=0$ and $n(\gamma,-1)=n(\gamma,0)$, hence this integral is an even integer, therefore
$$\exp\left( \frac{1}{2}\int_\gamma\frac{f^\prime(\zeta)}{f(\zeta)}\,\mathrm d\zeta \right)=1$$
This means $g$ is well-defined.
It is fairly standard to show that $g$ is continuous since $f^\prime/f$ is continuous wherever relevant.
It is also pretty clear that $g$ is analytic, since we can choose a local branch such that $\sqrt{\cdot}$ is analytic and $g(z)=\sqrt{f(z)}$.
Now the choice of $w_0$ gives two branches $g_+,g_-$ of $g$, so we get the function elements $(g_+,D_+),(g_-,D_-)$ where $D_+$ and $D_-$ are both copies of $D$.
Obviously, on the Riemann sphere $D$ is just the sphere removed two closed segments.
This is homeomorphic to the cylinder $S^1\times\mathbb R$.
So we want to figure out a way to glue this two cylinders corresonding to $D_+,D_-$ together along the branch cuts to make it the surface we wanted.
For $z_0\in (-1,0)\cup(1,\infty)$, we have
$$\lim_{z\to z_0^-}g_+(z)=\lim_{z\to z_0^+}g_-(z),\lim_{z\to z_0^+}g_+(z)=\lim_{z\to z_0^-}g_-(z)$$
where $z\to z_0^+$ denotes the limit of approaching $z_0$ from the upper half-plane and $z\to z_0^-$ is approaching from the lower half-plane.
This immediately tell us that we want to glue them together at the corresponding boundaries where the upper half-plane from one side is glued to the lower half-plane from the other side and leave $0,\pm 1,\infty$ alone.
This gives our Riemann surface $R$ (with the obvious conformal structure) which is a torus with $4$ points removed.
We see this via our geometrical intuition.
We will later see a computational technique to identify these surfaces.\\
Like in previous examples, we get some analytic functions with nice properties.
$g_\pm$ defines an analytic map $g:R\to\mathbb C$ and the inclusion $D_\pm\hookrightarrow\mathbb C\setminus{0,\pm 1}$ gives a covering map (also analytic) $\pi:R\to\mathbb C$.
Also, locally (hence globally) we have $g(p)^2=f\circ\pi(p)=\pi(p)^3-\pi(p)$.
    \section{The Theory of Covering Spaces}
\begin{definition}
    Suppose $\pi:\tilde{X}\to X$ is a covering map and $\gamma:[0,1]\to X$ is a path.
    A lift of $\gamma$ along $\pi$ is a path $\tilde{\gamma}:[0,1]\to\tilde{X}$ such that $\pi\circ\tilde{\gamma}=\gamma$.
\end{definition}
Obviously, lifts are not usually unique.
\begin{example}
    The function $\exp:\mathbb C\to\mathbb C_\star$ is a covering map.
    Consider the path $\gamma:[0,1]\to \mathbb C$ representing the unit circle, i.e. $\gamma(t)=e^{2\pi it}$, then both $t\mapsto 2\pi it$ and $t\mapsto 2\pi i+2\pi it$ are lifts of $\gamma$.
\end{example}
An interesting observation is that although we exhibited two different lifts, they do start at different points.
And it is indeed a correct intuition.
\begin{proposition}[Uniqueness of Lifts]
    Suppose $\tilde{\gamma}_1,\tilde{\gamma}_2$ are both lifts of $\gamma$ along a covering $\pi:\tilde{X}\to X$.
    If $\tilde{\gamma}_1(0)=\tilde{\gamma}_2(0)$ then $\tilde{\gamma}_1=\tilde{\gamma}_2$.
\end{proposition}
\begin{proof}
    Consider the set
    $$I=\{t\in[0,1]:\tilde{\gamma}_1(t)=\tilde{\gamma}_2(t)\}$$
    We shall show that it is both open and closed, which shows the proposition as $[0,1]$ is connected.
    It is obviously closed as $\tilde{\gamma}_1,\tilde{\gamma}_2$ are continuous and $[0,1]^2$ is Hausdorff.
    So it remains to show it is open.
    Let $t\in I$.
    As $\pi$ is a covering map, $\tilde{\gamma}_1(t)=\tilde{\gamma}_2(t)$ has a neighbourhood $\tilde{N}$ such that $\pi|_{\tilde{N}}$ is a homeomorphism.
    As $\gamma$ is continuous, there is $\delta>0$ such that $\gamma(t-\delta,t+\delta)\in N$.
    But for any $s$, $\pi\circ\tilde{\gamma}_1(s)=\gamma(s)=\pi\circ\tilde{\gamma}_2(s)$.
    So pick any $s\in (t-\delta,t+\delta)$, we have
    $$\tilde{\gamma}_1(s)=(\pi|_{\tilde{N}})^{-1}\circ\gamma(s)=\tilde{\gamma}_2(s)$$
    Therefore $s\in I$.
    This shows that $I$ is open, as desired.
\end{proof}
Now, even if $\pi$ is surjective, lifts may not exist.
\begin{example}[Counterexample]
    Consider $X=\mathbb C_\star,\tilde{X}=\mathbb R+i(-\pi,2\pi)$ and $\pi=\exp|_D$, but then we cannot lift the path $t\mapsto e^{2\pi it}$.
\end{example}
But note that the $\pi$ in the example above is not a regular covering map.
In fact, a lift does exist if $\pi$ is regular.
\begin{proposition}[Path-Lifting Lemma]
    Let $\pi:\tilde{X}\to X$ be a regular covering map and $\gamma:[0,1]\to X$ is a path.
    Suppose $\pi(\tilde{x})=\gamma(0)$, then there is a unique lift $\tilde\gamma$ of $\gamma$ such that $\tilde{\gamma}(0)=\tilde{x}$.
\end{proposition}
\begin{proof}
    Suffices to show the existence.
    Let
    $$I=\{t\in [0,1]:\gamma|_{[0,t]}\text{ can be lifted to some }\tilde\gamma\text{ with }\tilde{\gamma}(0)=\tilde{x}\}$$
    Again we will show that $I$ is both open and closed.\\
    To see $I$ is closed, consider a sequence $t_n\to\tau$ where $t_n\in I$ for all $I$.
    We shall show that $\tau\in I$.
    As $\pi$ is regular, there is some open $U\ni\gamma(\tau)$ such that
    $$\pi^{-1}(U)\cong\coprod_{\delta\in D}U_\delta$$
    for some set $D$.
    Throwing away finitely many terms we can assume $\gamma(t_n)\in U$ for any $t$, consequently $\tilde{\gamma}(t_n)$ are all in the same $U_\delta$.
    Set $\tilde{\gamma}(\tau)=(\pi|_{U_\delta})^{-1}\circ\gamma(\tau)$ extends $\tilde{\gamma}$ continuous to $\tau$, so $\tau\in I$.\\
    To see $I$ is open, let $\tau\in I$ and choose open $U\ni \gamma(\tau)$ such that
    $$\pi^{-1}(U)\cong\coprod_{\delta\in D}U_\delta$$
    for a set $D$.
    There is a unique $\delta$ such that $\tilde{\gamma}(\tau)\in U_\delta$.
    Choose $\epsilon>0$ such that $|t-\tau|<\epsilon\implies \gamma(t)\in U$.
    So we want to extend $\tilde{\gamma}$ via $\tilde\gamma(t)=(\pi|_{U_\delta})^{-1}\circ\gamma(t)$ for $|t-\tau|<\epsilon$, which works.
    Therefore $I$ is open, as required.
\end{proof}
\begin{definition}
    Let $X$ be a topological space and $\alpha,\beta:[0,1]\to X$ paths with $\alpha(0)=\beta(0),\alpha(1)=\beta(1)$.
    We say $\alpha,\beta$ are homotopic (or $\alpha\simeq\beta$) if there is a family of paths $(\alpha_s)_{s\in[0,1]}$ such that:\\
    1. $\alpha_0=\alpha,\alpha_s(1)=\beta$.\\
    2. $\alpha_s(0)=\alpha(0),\alpha_s(1)=\alpha(1)$ for any $s$.\\
    3. The map $(t,s)\mapsto\alpha_s(t)$ is continuous.
\end{definition}
\begin{definition}
    A topological space $X$ is simply connected if:\\
    1. $X$ is path-connected.\\
    2. Every pair of paths $\alpha,\beta:[0,1]\to X$ with the same endpoints are homotopic.
\end{definition}
\begin{remark}
    Let $D\subset\mathbb C$ be a convex domain, then the formula
    $$\alpha_s(t)=(1-s)\alpha(t)+s\beta(t)$$
    gives a homotopy between any two paths $\alpha,\beta$ with same endpoints.
\end{remark}
\begin{example}
    $\mathbb C$, the unit disk and half-plane are simply connected.
\end{example}
\begin{theorem}[Monodromy Theorem, aka Homotopy Lifting Lemma]\label{monodromy}
    Let $\pi:\tilde{X}\to X$ be a covering map and $\alpha,\beta$ in $X$ be such that:\\
    1. $\alpha\simeq\beta$ in $X$.\\
    2. There are lifts $\tilde\alpha$ of $\alpha$ and $\tilde\beta$ of $\beta$ along the covering.\\
    3. Every path $\gamma$ in $X$ with $\gamma(0)=\alpha(0)=\beta(0)$ has a lift $\tilde{\gamma}$ to $\tilde{X}$ with $\tilde{\gamma}(0)\tilde{\alpha}(0)=\tilde{\beta}(0)$.\\
    Then $\tilde{\alpha}\simeq\tilde{\beta}$.
    In particular, $\tilde{\alpha}\simeq\tilde{\beta}$.
\end{theorem}
\begin{proof}
    See Algebraic Topology.
\end{proof}
\begin{note}
    The requirements (ii) and (iii) are automatically satisfied if $\pi$ is regular.
\end{note}
    \section{The Monodromy Group; The Space of Germs}
\subsection{The Monodromy Group}
Let $\pi:\tilde{X}\to X$ be a regular covering map.
Pick a pasepoint $x_0\in X$.
For any choice of loop $\gamma:[0,1]\to X$ based at $x_0$ (that is $\gamma(0)=\gamma(1)=x_0$), we want to defines a permutation $\sigma_\gamma:\pi^{-1}(\{x_0\})\to\pi^{-1}(\{x_0\})$.
If you took Algebraic Topology, you should already know about this construction, but we'll do it again.
\begin{definition}
    Let $\tilde{x}\in\pi^{-1}(\{x_0\})$ and let $\tilde{\gamma}_{\tilde{x}}$ be the unique lift of $\gamma$ starting at $\tilde{x}$.
    Then $\pi(\tilde{\gamma}_{\tilde{x}}(1))=\gamma(1)=x_0$, so $\tilde{\gamma}_{\tilde{x}}(1)\in\pi^{-1}(\{x_0\})$.
    Therefore we define $\sigma_\gamma(\tilde{x})=\tilde{\gamma}_{\tilde{x}}(1)$.
\end{definition}
\begin{remark}
    1. The constant loop corresponds to the identity permutation.\\
    2. Let $\bar\gamma(t)=\gamma(1-t)$, then obviously $\sigma_{\bar\gamma}=\pi_{\gamma}^{-1}$, which precisely means that $\sigma_\gamma$ is a permutation.\\
    3. The previous two remarks hints that the set of all $\sigma_\gamma$ makes a subgroup of $\operatorname{Sym}(\pi^{-1}(\{x_0\}))$.
    We want to realise this group operation in an intuitive way.
    For $\alpha,\beta$ loops based at $x_0$, define their concatenation to be
    $$\alpha\cdot\beta=\begin{cases}
        \alpha(2t)\text{, for $t\in[0,1/2]$}\\
        \beta(2t-1)\text{, for $t\in[1/2,1]$}
    \end{cases}$$
    which is easily seen to be a well-defined loop based at $x_0$.
    The uniqueness of lifts then implies that
    $$(\widetilde{\alpha\cdot\beta})_{\tilde{x}_1}=\tilde{\alpha}_{\tilde{x}_1}\cdot\tilde{\beta}_{\tilde{\alpha}_{\tilde{x}_1}(1)}$$
    Therefore $\sigma_{\alpha\cdot\beta}=\sigma_\beta\sigma_\alpha$.
\end{remark}
\begin{definition}
    The group
    $$\{\sigma_\gamma|\gamma\text{ loop based at $x_0$}\}\le \operatorname{Sym}(\pi^{-1}(\{x_0\}))$$
    which is called the monodromy group of $\pi$.
\end{definition}
\begin{remark}
    1. By Theorem \ref{monodromy}, $\alpha\simeq\beta$ implies $\sigma_\alpha=\sigma_\beta$.\\
    2. One can easily show that the monodromy group is independent of the choice of basepoint (with the path-connectedness assumption, of course).
\end{remark}
\begin{example}
    Recall that $p_k:\mathbb C_\star\to\mathbb C_\star$ sending $z$ to $z^k$ is a regular covering map.
    Take basepoint $1$, then $\pi^{-1}(\{1\})$ consists of the $k^{th}$ roots of unity $\zeta_k^n=e^{2\pi in/k}$.
    Let $\gamma(t)=e^{2\pi it}$, then for each $n$, $\tilde{\gamma}_{\zeta_k^n}=\zeta_k^{n+1}$.
    Therefore $\sigma_\gamma(\zeta_k^n)=\zeta_k^{n+1}$.
    But it turns out every loop in $\mathbb C_\star$ is homotopic to $\gamma^n$ for some $n\in\mathbb Z$ (Algebraic Topology again!), therefore the monodromy group of $p_k$ is indeed the cyclic group of order $k$.
\end{example}
    \subsection{The Space of Germs}
Let $D\subset\mathbb C$ a domain.
\begin{definition}
    Let $(f,U)$ and $(g,V)$ be function elements on $D$.
    For any $z\in U\cap V$, write $(f,U)\equiv_z(g,V)$ if $f,g$ agree on a neighbourhood of $z$.
\end{definition}
Easily $\equiv_z$ is an equivalence relation.
\begin{definition}
    Let $(f,U)$ be a function element and $z\in U$.
    The equivalence class of $(f,U)$ under $\equiv_z$ is called the germ of $f$ at $z$ and is denoted by $[f]_z$.
\end{definition}
So two germs $[f]_z$ and $[g]_w$ are equal iff $z=w$ and $f=g$ on a neighbourhood of $z=w$.
We want to study all possible germs on a domain $D$.
\begin{definition}
    The space of germ over $D$ is
    $$\mathcal G=\{[f]_z:z\in D,(f,D)\text{ a function element with $z\in U$}\}$$
\end{definition}
Now we defined it as a set, it is natural to endow a topology on it.
For any function element $(f,U)$ on $D$, let $[f]_U=\{[f]_z:z\in U\}$.
\begin{lemma}
    The collection $\{[f]_U\}$ for $U$ open in $D$ is a basis of a topology on $\mathcal G$.
\end{lemma}
\begin{proof}
    Let $(f,U)$ and $(g,V)$ be function elements on $D$.
    For any $[h]_z\in [f]_U\cap [g]_V$, then $h$ agrees with $f,g$ on a neighbourhood $W$ of $z$, therefore $[h]_W\subset [f]_U\cap[g]_V$.
\end{proof}
This is the topology we want.
\begin{lemma}
    $\mathcal G$ is Hausdorff.
\end{lemma}
\begin{proof}
    Consider elements of $[f]_z,[g]_z\in\mathcal G$ with $[f]_z\neq [g]_z$.\\
    If $z\neq w$ then we can choose function elements $(f,U)\in [f]_z,(g,V)\in[g]_w$ such that $U\cap V=\varnothing$, therefore $[f]_U$ and $[g]_V$ are disjoint.\\
    If $z=w$, then we can choose an open neighbourhood $U$ such that $(f,U)\in[f]_z$ and $(g,U)\in [g]_z$.
    Unless $[f]_U\cap [g]_U=\varnothing$, there is a germ $[h]_z\subset [f]_U\cap[g]_U$.
    By the identity principle $f|_U=h|_U=g|_U$, which means $[f]_z=[h]_z=[g]_z$ due to the connectedness of $U$, contradiction.
\end{proof}
\begin{definition}
    Let $\mathcal G$ be the space of germs over a domain $D$.
    The forgetful map $\pi:\mathcal G\to D$ is defined by $\pi([f]_z)=z$.
\end{definition}
\begin{lemma}
    For each component $G\subset\mathcal G$, the restriction $\pi:G\to D$ is a covering map.
\end{lemma}
\begin{proof}
    Take an open $U\subset D$.
    Then the pre-image of $U$ has to be
    $$\pi^{-1}(U)=\bigcup_{(f,V)\text{ function element on }U}[f]_V$$
    which is open.
    So $\pi$ is continuous.\\
    For each open set in the form $[f]_U$, we have
    $$(\pi|_{[f]_U})^{-1}(z)=[f]_z$$
    which is a continuous inverse of $\pi|_{[f]_U}$.
    This shows that $\pi$ is a local homeomorphism, hence a covering map.
\end{proof}
Hence, by Lemma \ref{covering_conformal}, $\pi$ induces a well-defined conformal structure on $\mathcal G$ (well, on each of its connected components) such that $\pi$ is analytic.
Explicitly, the atlas we have in mind consists of charts $(\pi|_{[f]_U},[f]_U)$ across all the function elements $(f,U)$ on $D$.
\begin{definition}
    Let $\mathcal G$ be the space of germs on a domain $D$.
    The evaluation map $\mathcal E:\mathcal G\to\mathbb C$ is defined by $\mathcal E([f]_z)=f(z)$.
\end{definition}
In the chart $(\pi|_{[f]_U},[f]_U)$, we have
$$\mathcal E\circ(\pi|_{[f]_U})^{-1}(z)=\mathcal E([f]_z)=f(z)$$
Therefore $\mathcal E$ is analytic.
    \section{Uniqueness of Analytic Continuation; Gluing}
\subsection{Analytic Continuation Revisited}
The space of germs contains information about the class of analytic functions that agree on a neighbourhood of some given point.
Since we have seen that the space of germs admits a natural topological and analytical stucture, there should be some correlation between paths in this space and the analytic continuations along some paths in the original domain.
\begin{theorem}
    Let $(f,U),(g,V)$ be function elements on a domain $D\subset\mathbb C$ and $\gamma:[0,1]\to D$ be a path starting in $U$ and ending in $V$.
    Then $(f,U)\approx_\gamma (g,V)$ iff $\gamma$ lifts to some $\tilde{\gamma}$ in (a component of) $\mathcal G$ joining $[f]_{\gamma(0)}$ and $[g]_{\gamma(1)}$.
\end{theorem}
\begin{proof}
    Suppose $(f,U)\approx_\gamma(g,V)$, then we have $(f,U)=(f_1,U_1)\sim\cdots\sim (f_n,U_n)=(g,V)$ and a dissection $0=t_0<t_1<\cdots <t_{n-1}<t_n=1$ such that $\gamma([t_{i-1},t_i])\subset U_i$ for all $i\in\{1,\ldots,n\}$.
    Now define a lift $\tilde{\gamma}$ of $\gamma$ to $\mathcal G$ via $\tilde{\gamma}(t)=[f_i]_{\gamma(t)}$ for $t\in[t_{i-1},t_i]$.
    It is well-defined since $f_i|_{U_i\cap U_{i+1}}=f_{i+1}|_{U_i\cap U_{i+1}}$ as $(f_i,U_i)\sim (f_{i+1},U_{i+1})$ is a direct analytic continuation.
    Also observe that on $[t_{i-1},t_i]$ we have $\tilde{\gamma}=(\pi|_{[f_i]_{U_i}})^{-1}\circ\gamma$ which is continuous.
    Therefore $\tilde{\gamma}|_{[t_{i-1},t_i]}$ is continuous for all $i$, hence it is continuous.
    Now for each $t\in[t_{i-1},t_i]$, $\pi\circ\tilde{\gamma}(t)=\pi([f_i]_{\gamma(t)})=\gamma(t)$, so $\tilde{\gamma}$ does lift $\gamma$.
    Easily $\tilde{\gamma}(0)=[f_1]_{\gamma(0)}=[f]_{\gamma(0)}$ and $\tilde{\gamma}(1)=[f_n]_{\gamma(1)}=[g]_{\gamma(1)}$ by construction, as required.\\
    Conversely, suppose such $\tilde\gamma$ exists, then every point $\tilde{\gamma}(t)$ has a neighbourhood $[f_t]_{U_t}$ where $(f_t,U_t)$ is a function element on $D$ and each $U_t$ is a disk.
    Compactness of $[0,1]$ we can choose a finite collection of function elements $(f_1,U_1),\ldots,(f_n,U_n)$ among them and a dissection $0=t_0<t_1<\cdots <t_{n-1}<t_n=1$ such that $\tilde{\gamma}([t_{i-1},t_i])\subset [f_i]_{U_i}$.
    Then as $\tilde\gamma$ is a lift of $\gamma$, $\gamma([t_{i-1},t_i])\subset U_i$ for any $i$.
    Also for any $i$, $[f_{i-1}]_{\gamma(t_{i-1})}=\tilde{\gamma}(t_{i-1})=[f_i]_{\gamma(t_{i-1})}$, therefore $f_{i-1}$ and $f_i$ agrees on a neighbourhood of $\gamma(t_{i-1})\in U_{i-1}\cap U_i$.
    But since $U_{i-1},U_i$ are disks, $U_{i-1}\cap U_i$ is connected and hence $f_{i-1}=f_i$ on $U_{i-1}\cap U_i$ by the identity principle.
    Therefore it indeed gives the desired analytic continuation.
\end{proof}
\begin{corollary}
    Let $\mathcal F$ be a complete analytic function on a domain $D\subset\mathbb C$, then
    $$\mathcal G_{\mathcal F}=\bigcup_{(f,U)\in\mathcal F}[f]_U$$
    is a path component of $\mathcal G$.
\end{corollary}
Therefore complete analytic functions on a domain $D\subset\mathbb C$ are equivalent to Riemann surfaces equipped with covering maps defined by the restriction of the forgetful map.
\begin{definition}
    The component $\mathcal G_{\mathcal F}$ is the Riemann surface associated to $\mathcal F$.
\end{definition}
    \subsection{The Classical Monodromy Theorem}
\begin{theorem}[Classical Monodromy Theorem]
    Let $D\subset\mathbb C$ be a domain and $(f,U)$ is a function element in $D$ that can be analytically continued along any path in $D$ starting in $U$.
    If $(f,U)\approx_\alpha (g_1,V)$ and $(f,U)\approx_\beta(g_2,V)$ and $\alpha\simeq\beta$, then $g_1=g_2$ on $V$.
\end{theorem}
Therefore analytically continuing along a path only depends on the homotopy class of the path of continuation.
\begin{proof}
    Let $\tilde{\alpha},\tilde{\beta}$ be the lifts of $\alpha,\beta$ to $\mathcal G$ such that $\tilde{\alpha}(0)=[f]_{\alpha(0)}=\tilde{\beta(0)}$.
    As $\alpha\simeq\beta$, we have $\tilde{\alpha}\simeq\tilde{\beta}$ by Theorem \ref{monodromy}.
    Hence $\tilde{\alpha}(1)=\tilde{\beta}(1)$, which means $[g_1]_{\alpha(1)}=[g_2]_{\beta(1)}$, so $g_1,g_2$ coincides on some neighbourhood of $\alpha(1)=\beta(1)$, which implies $g_1=g_2$ on $V$ by the identity principle.
\end{proof}
\begin{corollary}
    Let $D$ be a simply connected domain and $(f,U)$ a function element on $D$.
    If $(f,U)$ can be analytically continued along every path in $D$ starting in $U$, then $(f,U)$ extends to an analytic function $f:D\to\mathbb C$.
\end{corollary}
\begin{proof}
    Immediate.
\end{proof}
Nice.
    \subsection{Gluing Riemann Surfaces}
When we were studying the $k^{th}$ roots, we constructed a Riemann surface $R_k$ equipped with analytic $\pi,g$ such that
\[
    \begin{tikzcd}
        R_k\arrow[swap]{dr}{\pi}\arrow{r}{g}&\mathbb C_\star\arrow{d}{p_k}\\
        &\mathbb C_\star
    \end{tikzcd}
\]
commutes.
We also observed that this diagram can be compactified by resolving the removable singularities
\[
    \begin{tikzcd}
        \hat{R}_k\arrow[swap]{dr}{\hat\pi}\arrow{r}{\hat{g}}&\mathbb C_\infty\arrow{d}{\hat{p}_k}\\
        &\mathbb C_\infty
    \end{tikzcd}
\]
How do we do this in general?
The answer is via gluing.
\begin{definition}
    Let $X,Y$ be topological spaces and with subspaces $X'\subset X,Y'\subset Y$ and a homeomorphism $\Phi:X'\to Y'$.
    The result of gluing $X,Y$ along $\Phi$ is the topological space $Z=(X\sqcup Y)/\sim$ where $\sim$ is the minimal equivalence relation such that $x\sim\Phi(x)$ for all $x\in X'$.
    It is sometimes denoted by $X\cup_{\Phi}Y$ or $X\cup_{X'}Y$ if the homeomorphism $\Phi$ is understood.
\end{definition}
We need to understand how gluing gives rise to a new Riemann surface in the case where $X,Y$ are Riemann surfaces.
\begin{proposition}
    Let $R_1,R_2$ be Riemann urfaces and $S_j\in R_j$ are non empty, connected and open subsets and $\Phi:S_1\to S_2$ is a conformal equvalence of Riemann surfaces, then there is a unique conformal stucture on $R=R_1\cup_\Phi R_2$ such that the inclusions $i_j:R_j\to R$ are analytic.
    In particular, if $R$ is Hausdorff, then it is a Riemann surface.
\end{proposition}
\begin{proof}
    Consider the family of charts $(\phi_j\circ i_j^{-1},i_j(U_j))$ where $(\phi_j,U_j)$ is a chart on $R_j$.
    The transition functions are then either transition functions of $R_j$ or $\phi_2\circ i_2^{-1}\circ i_1\circ\phi_1^{-1}=\phi_2\circ\Phi\circ\phi_1^{-1}$ which is analytic as $\Phi$ is.
    This induces a conformal structure on $R$.\\
    For uniqueness, suppose $(\phi_j,U_j)$ is a chart on $R_j$ and $(\psi,V)$ is a chart in another conformal stucture on $R$ such that the condition holds, then $\psi\circ i_j\circ\phi_j^{-1}$ is analytic since $i_j$ is.
    This means that $\phi_j\circ i_j^{-1}$ has analytic transition function with all charts
    So by maximality the two conformal structures are equal.\\
    It is quite obvious that $R$ is connected.
    So if we assume further that $R$ is Hausdorff, then $R$ is a Riemann surface.
\end{proof}
\begin{example}[Non-example]
    Take $R_1=R_2=\mathbb C,S_1=S_2=\mathbb C_\star$, then $R=\mathbb C\cup_{\operatorname{id}_{\mathbb C_\star}}\mathbb C$ is not Hausdorff.
\end{example}
\begin{example}
    Let $R_1=R_2=\mathbb C$ and $S_1=S_2=\mathbb C_\star$ and let $\Phi:\mathbb C_\star\to\mathbb C_\star$ be the inversion $z\mapsto 1/z$.
    Then $R=\mathbb C\cup_\Phi\mathbb C$ is obviously Hausdorff hence is a Riemann surface by the preceding proposition.
    One can also see easily that $R$ is compact.
    Easily, one can see that $R$ is exactly the Riemann sphere $\mathbb C_\infty$.
\end{example}
    \section{More Gluing; Branching}
\subsection{A More Detailed Gluing Example}
We have seen how to glue two copies of $\mathbb C$ to form the Riemann sphere $\mathbb C_\infty$.
We are gonna try something more interesting:
The Riemann surface $R$ associated with $w^2=z^3-z$.
Topologically, we know it is homeomorphic to the torus with $4$ points removed.
We already know its conformal structure, but we are also interested in the analytic functions on it.
We want to compactify $R$ by adding $4$ points.
But we already know (from example sheet) that we can extend $R$ to the Riemann surface $R_1=\{(z,w)\in\mathbb C:w^2=z^3-z\}$, so it remains to compactify $R$ by adding the point $\infty$.
The idea for this is to perform a change of coordinate sending $\infty$ to a finite point.
Consider $u=1/z$ and $v=z/w$ (the method to obtain it will be covered in Algebraic Geometry), so $z=1/u$ and $w=z/v=1/{uv}$, so the original equation becomes $u=v^2(1-u^2)$.
The set $R_2=\{(z,w)\in\mathbb C^2:u=v^2(1-u^2)\}$ is then a Riemann surface where the atlas is defined by the restriction of the projections $\pi,\tau$ to the first and second coordinate.
One can do some technical work to check that it works and $\pi,\tau$ are analytic functions under this conformal structure.
We are going to glue $R_1$ and $R_2$ together to get a compact Riemann surface.
The gluing map is given by $\Phi(z,w)=(u,v)=(1/z,z/w)$ and $\Phi^{-1}(u,v)=(z,w)=(1/u,1/(uv))$ between $S_1=R_1\setminus\{(0,0),(\pm 1,0)\}$ and $S_2=R_1\setminus\{(0,0)\}$.
These are conformal equivalences as the coordinate functions are holomorphic and the atlases are defined byb coordinate projection.
Glue them together gives $\hat{R}=R_1\cup_\Phi R_2$ which is connected.
Before we check that it is Hausdorff, we want to first note that $\hat\pi_1:R_1\to\mathbb C_\infty$ via $(z,w)\mapsto z$ and $\hat{\pi}_2:R_2\to\mathbb C_\infty$ via $(u,v)\mapsto 1/u$ satisfy $\hat\pi_1=\hat\pi_2\circ\Phi$.
So they define a continuous function $\hat\pi$ on $\hat{R}$ (which is automatically meromorphic once we can show that $\hat{R}$ is Hausdorff hence is a Riemann surface).
Let $i_1:R_1\to\hat{R}$ and $i_2:R_2\to\hat{R}$ are the respective inclusions, then to see $\hat{R}$ is Hausdorff, it suffices to seperate $\hat{R}\setminus i_2(R_2)=\{i_1(0,0),i_1(\pm 1,0)\}$ from $\hat{R}\setminus i_1(R_1)=\{i_2(0,0)\}$.
This can be done by simply taking $\hat\pi^{-1}(D(0,2))$ and $\hat\pi^{-1}(\mathbb C_\infty\setminus\bar{D}(0,2))$.\\
Now to see $\hat{R}$ is compact, we shall prove it is sequential compact.
Choose a sequence $(p_n)_{n\ge 0}$ in $\hat{R}$.
Now if we, via restricting to a subsequence, have $|\hat\pi(p_n)|\le M$ for all $n$, then a further subsequence of $\hat\pi(p_n)$ converges to some $z_0$.
But there are at most two $w_0$ such that $i_1(z_0,w_0)\in\hat{R}$, so $(p_n)$ has a further subsequence that converges to $i_1(z_0,w_0)$ for one of those $w_0$.
Otherwise $|\hat\pi(p_n)|\to\infty$, but $p_\infty=i_2(0,0)$ is the only point of $\hat{R}$ whose image under $\hat\pi$ is $\infty$, therefore $p_n\to p_\infty$ as $n\to\infty$.
Either way there is a convergent subsequence, so $\hat{R}$ is sequential compact hence compact.\\
In summary, we obtained a compact Riemann surface $\hat{R}$ associated with $w^2=z^3-z$.
Just like $\pi$ extends to a meromorphic function $\hat\pi$, we can extend $g$ to a meromorphic $\hat{g}$ in the same way.
    \subsection{Branching}
As observed, when we compactify a surface, something that used to be a covering map no longer has that property.
\begin{example}
    The map $p_k:\mathbb C_\star\to\mathbb C_\star$ via $z\mapsto z^k$ is a covering map, but $\hat{p}_k:\mathbb C_\infty\to\mathbb C_\infty$ is not (if $k\ge 2$).
\end{example}
\begin{definition}
    let $f:R\to S$ be analytic.
    For $p\in R$, recall that we can find charts that put $f$ into a standard local form $\psi\circ f\circ\phi^{-1}(z)=z^n$ near $p$ for some $n\in\mathbb Z_{\ge 0}$ by Proposition \ref{local_p_k}.
    The integer $n$ does not depend on the choice of charts (as we can alternatively define it to be the size of $f^{-1}(\{f(p)\})$, and is called the multiplicity $m_f(p)$ of $f$ at $p$.
\end{definition}
Most points have multiplicity $1$, the remaining ones are especially interesting.
\begin{definition}
    If $m_f(p)>1$, then $p$ is called the ramification point of $f$ and $f(p)$ is called a branch point of $f$.
    In this case, $m_f(p)$ is sometimes also called the ramification index.
\end{definition}
\begin{example}
    The map $\hat{p}_k:\mathbb C_\infty\to\mathbb C_\infty$ via $z\mapsto z^k$ and $0\mapsto 0,\infty\mapsto\infty$.
    Then the ramification points are $z=0,\infty$ and the branch points are $w=0,\infty$.\\
    More generally, for polynomials $f:\mathbb C_\infty\to\mathbb C_\infty$, $f(z)=a_dz^d+\cdots+a_0$ ($a_d\neq 0$) for $z\neq\infty$ and $f(\infty)=\infty$, changing the variable $w=1/z$ gives
    $$\frac{1}{f(z)}=\frac{1}{a_dz^d+\cdots+a_0}=\frac{w^d}{a_d+\cdots+a_0w^d}=w^dg(w)$$
    where $g$ is (locally) analytic and nonzero, hence $m_\infty(f)=d$.
\end{example}
\begin{remark}
    Let $f:R\to\mathbb C$ be an analytic function, $p\in R$ and $(\phi,U)$ a chart containing $p$.
    Then
    $$F(z)=f\circ\phi^{-1}(z)=(z-z_0)^mg(z),z_0=\phi(p)$$
    where $m=m_f(p)$ by definition.
    So $F^\prime(z)=(mg(z)+(z-z_0)g^\prime(z))(z-z_0)^{m-1}$.
    If $m=1$, then $F^\prime(z_0)=g(z_0)\neq 0$.
    But if $m>1$, then $F^\prime(z_0)=mg(z_0)(z_0-z_0)^{m-1}=0$.
    Therefore the ramification points are exactly the zeros of $F^\prime$.
\end{remark}
\begin{lemma}
    If $f:R\to S$ and $g:S\to T$ are analytic, then $m_{g\circ f}(p)=m_g(f(p))m_f(p)$ for any $p\in R$.
\end{lemma}
\begin{proof}
    Find the respective local coordinates such that $f$ corresponds to $z\mapsto z^{m_f(p)}$ and $g$ corresponds to $w\mapsto w^{m_g(f(p))}$, then in those local coordinates $g\circ f$ is $z\mapsto(z^{m_f(p)})^{m_g(f(p))}=z^{m_g(f(p))m_f(p)}$.
    The equality follows.
\end{proof}
    \section{The Valency Theorem; Euler Characteristic; The Riemann-Hurwitz Theorem}
\subsection{The Valency Theorem}
We want to relate the branching data of an analytic map and the topology of a compact Riemann surface.
\begin{theorem}[Valency Theorem]
    Suppose $f:R\to S$ is a non-constant analytic map between compact Riemann surfaces $R,S$, then the function $n:S\to\mathbb N$ defined by
    $$n(q)=\sum_{p\in f^{-1}(\{q\})}m_f(p)$$
    is constant on $S$.
\end{theorem}
\begin{proof}
    By the identity principle, $f^{-1}(\{q\})$ is finite, therefore $n$ is well-defined.
    As $S$ is connected, it suffices to show that $n$ is locally constant.
    Let $q_0\in S$ and $f^{-1}(q_0)=\{p_1,\ldots,p_k\}$.
    Our goal is to find local coordinates about the $p_i$ and $q_0$ such that $f$ is a power map.
    Let $(\psi,V)$ be a chart about $q_0$ such that $\psi(q_0)=0$, then there exists disjoint charts $(\phi_1,U_1),\ldots,(\phi_k,U_k)$ such that $p_i\in U_i$ and $\psi\circ f\circ\phi_i^{-1}(z)=z^{m_f(p_i)}$.\\
    Let $U=\bigcup_iU_i$, then $U$ is open, so $R\setminus U$ is closed subset of $R$, hence is compact.
    So $K=f(R\setminus U)$ is compact and hence closed as $S$ is Hausdorff.
    Set $V'=V\setminus K$.
    Now quite obviously $f^{-1}(V')\subset U$.
    Set $U_i'=U_1\cap f^{-1}(V')$, then $f^{-1}(V')=\bigcup_iU_i'$, then in the charts $(\phi_i,U_i')$ and $(\psi,V')$, $f$ takes the form of power maps, therefore $n(q)=n(q_0)$ for any $q\in V'$.
\end{proof}
\begin{definition}
    This constant $n$ is called the valency or degree $\deg f$ of $f$.
\end{definition}
\begin{example}
    If $f$ is a polynomial, then the degree of $f$ equals the polynomial degree of $f$.
\end{example}
\begin{corollary}[Fundamental Theorem of Algebra]
    Any nonconstant polynomial of degree $d$ has exactly $d$ zeros in $\mathbb C$.
\end{corollary}
\begin{proof}
    Extend it analytically to $\mathbb C_\infty\to\mathbb C_\infty$.
\end{proof}
    \subsection{Euler Characteristic}
\begin{definition}
    Let $S$ be a compact Riemann surface.
    A topological triangle in $S$ is a continuous embedding $\Delta\hookrightarrow S$ where $\Delta$ is a closed (non-degenerate) triangle in the plane $\mathbb R^2$.
    A triangulation of $S$ is a finite collection of topological triangles $\{\Delta_i\}$ on $S$ such that:\\
    1. $\bigcup_i\Delta_i=S$.\\
    2. If $i\neq j$ then $\Delta_i\cap\Delta_j$ is either empty, a common vertex, or a common edge.\\
    3. Each edge is contained in exactly two triangles.
\end{definition}
\begin{definition}
    The Euler characteristic of a triangulation of $S$ is $\chi=V-E+F$, where $V$ is the number of vertices, $E$ is the number of edges and $F$ is the number of triangles.
\end{definition}
\begin{lemma}
    1. Every compact Riemann surface $S$ has a triangulation.\\
    2. $\chi$ does not depend on the triangulation we choose.
\end{lemma}
\begin{proof}
    Omitted.
\end{proof}
\begin{definition}
    The Euler characteristic $\chi(S)$ of $S$ is the Euler characteristic of any of its triangulations.
\end{definition}
\begin{example}
    $\chi(\mathbb C_\infty)=4-6+4=2$ by identifying it with a regular tetrahedron.
    $\chi(\mathbb C/\Lambda)=0$ by attempting to triangulate its representation as a quotient space of $[0,1]^2$.
\end{example}
Turns out, every compact Riemann surface is homeomorphic to an $g$-torus $\Sigma_g$ for some $g$.
Moreover, $\chi(\Sigma_g)=2-2g$.
Therefore $\chi(S)$ determines a compact Riemann surface $S$ up to homeomorphism.
    \subsection{The Riemann-Hurwitz Theorem}
\begin{theorem}[Riemann-Hurwitz]
    Let $f:R\to S$ be any non-constant analytic map of compact Riemann surfaces, then
    $$\chi(R)=\deg(f)\chi(S)-\sum_{p\in R}(m_f(p)-1)$$
\end{theorem}
\begin{remark}
    As $R$ is compact, the sum only has finitely many nonzero terms.
\end{remark}
\begin{proof}[Sketch of proof]
    As in the proof of the valency theorem, each $q\in S$ has a ``power neighborhood'' $U$ where $f$ restricts to a union of power maps on $f^{-1}(U)$.
    By compactness, there is a finite open cover $\{U_1,\ldots,U_k\}$ of $S$ where each $U_i$ is a ``power neighbourhood'' of $f$.
    In particular, the number of branch points is finite.
    We can subdivide a triangulation on $S$ so that we can eventually reach a triangulation such that each triangle has at most $1$ branch point.
    We can further subdivide such that each branch point is a vertex.
    Continue to subdivide so that each triangle is contained in some $U_i$.
    Now the preimage of this eventual triangulation forms a triangulation of $R$.
    Let $n=\deg f$ and $V_R,E_R,F_R,V_S,E_S,F_S$ are exactly what you think they mean.
    Then, intuitively, $F_R=nF_S,E_R=nE_S$ while
    $$|f^{-1}(\{q\})|=n-\sum_{p\in f^{-1}(\{q\})}(m_f(p)-1)$$
    Summing up,
    $$V_R=nV_S-\sum_{q\in S}\sum_{p\in f^{-1}(\{q\})}(m_f(p)-1)=nV_S-\sum_{p\in R}(m_f(p)-1)$$
    which implies the identity.
\end{proof}
    \section{Applications of Riemann-Hurwitz}
\subsection{Immediate Consequences}
We can rearrange Riemann-Hurwitz to
$$2g_R-2=n(2g_S-2)+\sum_{p\in R}(m_f(p)-1)$$
where $g_R,g_S$ are the genera of $R,S$ respectively and $n$ is the degree of $f$.
We can use it to calculate genera of Riemann surfaces.
\begin{example}
    Consider the compactification $\hat{R}$ of the Riemann surface associated with $w=\sqrt{z^3-z}$ equipped with a meromorphic function $\hat{\pi}:\hat{R}\to\mathbb C_\infty$.
    We shall calculate the genus of $\hat{R}$ by Riemann-Hurwitz.
    Take $f=\hat\pi$, then $n=\deg\hat\pi=2$ by valency theorem.
    The branch points are $0,\pm 1,\infty$ and each of them has exactly $1$ preimage (branch points like this are called ``totally ramified''), hence has multiplicity $2$.
    Plugging these information into Riemann-Hurwitz yields
    $$2g_{\hat{R}}-2=2(0-2)+4(2-1)\implies g_{\hat{R}}=1$$
    which is consistent with the fact that $\hat{R}$ is topologically a torus.
\end{example}
\begin{remark}
    The correction term $\sum_{p\in R}(m_f(p)-1)$ is always even.
    This is obvious but quite useful from time to time.
    We obtained our compact Riemann surface in the above example from gluing $\hat{R}=R_1\cup_\Phi R_2$.
    Imagine we know nothing about $\hat\pi^{-1}(\{\infty\})\subset R_2$ and write the correction term as
    $$3(2-1)+\sum_{p\in\hat\pi^{-1}(\infty)}(m_{\hat\pi}(p)-1)=3+C,C=\sum_{p\in\hat\pi^{-1}(\infty)}(m_{\hat\pi}(p)-1)$$
    If $\infty$ is not a ramification point, then $C=0$, which gives an odd correction point which is impossible.
    Therefore it has to be the case that $\infty$ is a ramification point (which has to be totally ramified) and $C=1$.
    Therefore we can go directly from there to obtain $g_{\hat{R}}=1$.
    Hence, when $\deg f=2$, then we can obtain the branching at $\infty$ for free from this parity argument.
\end{remark}
\begin{remark}
    In the case when $f$ is a covering map (aka unramified), then the correction term vanished, therefore $g_R-1=n(g_S-1)$.
    There are three cases:\\
    (i) If $g_S=0$, then $g_R-1<0$, which means actually $g_R=0,n=1$.
    But degree $1$ maps have to be conformal equivalences, therefore $R\cong S$.
    Also genus $0$ surfaces are simply the Riemann sphere, so $f$ is just a M\"obius transformation.\\
    (ii) If $g_S=1$, then necessarily $g_R=1$, but then $n$ is not restricted.\\
    (iii) If $g_S>1$, then either $g_R=g_S$ and $n=1$ (in which case $f$ is a conformal equivalence) or $g_R>g_S$ and $n>1$.
\end{remark}
\begin{example}
    Consider the family of lattices $\Lambda_n=\langle n,i\rangle\le\mathbb C$ for $n\in\mathbb Z_{>0}$.
    Then $\Lambda_n\le \lambda_1$ for all $n$ which induces a covering map $\mathbb C/\Lambda_n\to\mathbb C/\Lambda_1$ which has degree $n$.
    Therefore when $g_S=1$ there is truly no restriction on the degree of $f$.
\end{example}
    \subsection{Higher-Genus Surfaces}
We want to construct Riemann surfaces with genus $n>1$.
\begin{example}
    Consider the Fermat curve of degree $d$:
    $$F_d'=\{(x,y)\in\mathbb C^2:x^d+y^d=1\}$$
    the understanding of whose rational point, incidentally, is Fermat's Last Theorem.
    This is obviously not we are interested in here.
    As usual we want to make it a Riemann surface.
    The coordinate projection $\pi_x:(x,y)\mapsto x$ has local inverse $\pi_x^{-1}(x_0)=(x_0,\sqrt[d]{1-x_0^d})$ which exists and is continuous in a neighbourhood of any $x_0$ unless $x_0$ is a $d^{th}$ root of unity, i.e. $x=\zeta_d^i$ for some $i$ where $d=\exp(2\pi i/d)$, which happens iff $y_0=0$ (where $(x_0,y_0)\in F_d'$).
    Symmetrically, $\pi_y$ provides charts on $F_d'\setminus\{(0,\zeta_d^i)\}$.
    These charts cover everything as the points where $\pi_x$ and $\pi_y$ don't work are distinct.
    The non-trivial transition functions are $\pi_y\circ\pi_x^{-1}(x)=\sqrt[d]{1-x^d},\pi_x\circ\pi_y^{-1}(y)=\sqrt[d]{1-y^d}$ which are analytic where they need to be.
    It is quite obvious that $F_d'$ is Hausdorff since it inherits its topology from $\mathbb C^2$.
    To see it is connected, consider
    $$D=\mathbb C\setminus\left(\bigcup_{i=1}^d\{t\zeta_d^i:t\ge 1,i\in\{0,\ldots,d-1\}\}\right)$$
    Now there are well-defined branches of $y=\sqrt[d]{1-x^d}$ in $D$ which can be extend continuously to $\zeta_d^i$.
    Let $(x_0,y_0)\in F_d'$.
    Suppose $x_0\in D$, then we can choose a branch of $y(x)=\sqrt[d]{1-x^d}$ such that $y(x_0)=y_0$.
    Let $\gamma$ be any path in $D$ from $x_0$ to $1$, then $\tilde\gamma(t)=(\gamma(t),y(\gamma(t)))$ is a path joining $(x_0,y_0)$ to $(1,0)$.
    As for the $d$ rays in $F_d'\setminus D$, we can also find a branch locally near that and join it to something nearby that is in $D$ (possible as $D$ is dense in $\mathbb C$).
    Therefore $F_d'$ has to be path-connected.
    Consequently it is indeed a Riemann surface.\\
    We can compactify by gluing $F_d=F_d'\cup_\Phi F_d''$.
    The details can be found from example sheet, where we also extend the covering map to a meromorphic $\hat\pi_x:F_d\to\mathbb C_\infty$ with $\deg\hat\pi_d=d$.
    Also $|\hat\pi_x^{-1}(\infty)|=d$, so $\infty$ is not a ramification point.
    The ramification points are then $\{(\zeta_d^i,0),i\in\{0,\ldots,d-1\}\}$ and all of them have multiplicity $d$.
    Riemann-Hurwitz then gives
    $$2g_{F_d}-2=d(0-2)+d(d-1)\implies g_{F_d}=\frac{(d-1)(d-2)}{2}$$
\end{example}
The conclusion is then there does exist Riemann surfaces with arbitrarily large genus.
    \section{Rational and Periodic Functions}
\subsection{Rational Functions}
We want to study meromorphic functions in the Riemann sphere, which are simply the rational functions.
\begin{proposition}
    Every meromorphic function $f:\mathbb C_\infty\to\mathbb C_\infty$ on the Riemann sphere is a rational function, that is we can write
    $$f(z)=c\frac{(z-a_1)\cdots (z-a_m)}{(z-b_1)\cdots (z-b_n)}$$
    for $c,a_i,b_i\in\mathbb C$.
\end{proposition}
Given an $f$ in this form, we can assume WLOG that $a_i\neq b_j$ for all $i,j$.
\begin{proof}
    The case that $f$ is constant is trivial.
    Replacing $f\to 1/f$ necessary, then we can assume WLOG that $f(\infty)\neq\infty$.
    Let $b_1,\ldots,b_{n'}$ be poles of $f$ (there are finitely many as the Riemann sphere is compact).
    Then by assumption $b_i\neq\infty$ for all $i$.
    The Laurent series about $b_j$ then has the form
    $$f(z)=\sum_{l=-k_j}^\infty c_{j,l}(z-b_j)^l,c_{-k_j,j}\neq 0$$
    then $k_j$ is the order of the pole.
    Let
    $$Q_j(z)=\sum_{l=-k_j}^{-1} c_{j,l}(z-b_j)^l$$
    be the principal part of the series and let
    $$g(z)=f(z)-\sum_{j=1}^{n'}Q_j(z)$$
    Then $g$ is entire and has a removable singularity at $\infty$, hence it has to be constant, which means $f$ has to be rational.
\end{proof}
\begin{remark}
    In th proof, $f(\infty)\in\mathbb C$ means $m\le n$, and in this case we have $\deg f=\sum_jk_j=n$, therefore in general the degree of a rational function is $\max\{m,n\}$.
\end{remark}
    \subsection{Simply Periodic Functions}
Our next goal is to classify meromorphic functions on other Riemann surfaces.
Many Riemann surfaces we have described in the form $R=D/\sim$ where $D$ is a domain and $\sim$ is an equivalence relation.
This is useful in the sense that functions on $R$ are automatically periodic on $D$ with respect to $\sim$.
\begin{definition}
    Let $f:\mathbb C\to\mathbb C_\infty$ be meromorphic,
    A period of $f$ is a complex number $\omega\in\mathbb C$ such that $f(z+\omega)=f(z)$ for all $z\in\mathbb C$.
\end{definition}
Note that the periods of $f$ forms a additive subgroup $\Omega\le\mathbb C$.
\begin{lemma}
    Let $\Omega$ be the set of periods of a meromorphic function $f$ on $\mathbb C$, then one of the following holds:\\
    (i) $\Omega=\{0\}$.\\
    (ii) $\Omega=\langle \omega\rangle\cong\mathbb Z$ for $\omega\neq 0$.\\
    (iii) $\Omega=\langle w_1,w_2\rangle\cong\mathbb Z^2$ where $\omega_1,\omega_2\neq 0,\omega_1/\omega_2\notin\mathbb R$.\\
    (iv) $\Omega=\mathbb C$.
\end{lemma}
\begin{proof}
    See example sheet.
\end{proof}
\begin{definition}
    A meromorphic function $f$ on $\mathbb C$ whose group of periods contains $\langle\omega\rangle\cong\mathbb Z,\omega\neq 0$ is called simply periodic.
\end{definition}
\begin{example}
    $\exp$ has $\Omega=\langle 2\pi i\rangle$.
\end{example}
Observe that $\exp$ is also a covering map.
In fact,
\begin{proposition}\label{simply_periodic}
    If $f$ is a meromorphic function on $\mathbb C$ and the periods of $f$ contains an infinite cyclic subgroup $\langle\omega\rangle$, then there is a unique meromorphic function $\bar{f}$ on $\mathbb C_\star$ such that $f(z)=\bar{f}\circ\exp(2\pi iz/\omega)$.
\end{proposition}
\begin{proof}
    Uniqueness is trivial.
    For existence, we choose any branch of $\log$ and define $\bar{f}(w)=f(\omega\log(w)/(2\pi i))$ which does satisfy $\bar{f}\circ\exp(2\pi iz/\omega)=f$.
    It remains to show that $\bar{f}$ is well-defined.
    Suppose we have chosen a different branch of $\log$, then the function we obtained instead would be $\hat{f}=f(\omega(\log w+2\pi in)/(2\pi i))$ for some $n\in\mathbb Z$.
    But this is just $\bar{f}(w)$ since $n\omega$ is a period of $f$.
\end{proof}
Therefore simply periodic functions are in one-to-one correspondence to functions on $\mathbb C_\star$.
This is natural since $\mathbb C_\star\cong\mathbb C/\langle\omega\rangle$ conformally via $z\mapsto\exp(2\pi iz/\omega)$.
    \subsection{Doubly Periodic Functions}
\begin{definition}
    A meromorphic function $f$ on $\mathbb C$ whose period contains $\langle\omega_1\rangle\oplus\langle\omega_2\rangle\cong\mathbb Z^2$ is said to be doubly periodic or elliptic.
\end{definition}
In this case, $\Omega$ is a lattice $\Lambda$, so $f$ depends on a meromorphic function on $\mathbb C/\Lambda$.
\begin{proposition}
    If $f$ is a meromorphic function on $\mathbb C$ and the periods of $f$ contain a lattice $\Lambda$, then there is a unique meromorphic function $\bar{f}$ on the complex torus $\mathbb C/\Lambda$ such that $f(z)=\bar{f}\circ\pi(z)$ where $\pi:\mathbb C\to\mathbb C/\Lambda$ is the quotient map.
\end{proposition}
\begin{proof}
    Same as Proposition \ref{simply_periodic} by observing that $\pi$, as a covering map, admits a local analytic inverse.
\end{proof}
So the elliptic functions on $\mathbb C$ with $\Lambda\subset\Omega$ corresponds to meromorphic functions on $\mathbb C/\Lambda$.
This means that we can apply our study in compact Riemann surfaces in doubly periodic functions.
\begin{corollary}
    There is no doubly periodic analytic functions on $\mathbb C$ other than constant maps.
\end{corollary}
\begin{proof}
    Immediate.
\end{proof}
Also, since we can view a doubly periodic function $f$ as an analytic maps $\bar{f}$ in the compact surface $\mathbb C/\Lambda$, we can make sense of its degree by writing $\deg f=\deg\bar{f}$.
\begin{corollary}\label{degree_doubly_periodic}
    If $f:\mathbb C\to\mathbb C_\infty$ is doubly periodic and non-constant, then $\deg f\ge 2$.
\end{corollary}
\begin{proof}
    If $\deg f=1$, then $f$ is unramified so Riemann-Hurwitz gives $0=1(-2)+0=-2$, contradiction.
\end{proof}
For a lattice $\Lambda=\langle\omega_1,\omega_2\rangle$, we construct the period parallelogram as the parallogram $P$ with vertices $z_0,z_0+\omega_1,z_0+\omega_2,z_0+\omega_1\omega_2$ for some fixed $z_0\in\mathbb C$ (usually $0$).
So $f$ is determined by its value on $P$.
\begin{proof}[Alternative proof of Corollary \ref{degree_doubly_periodic}]
    Choose $z_0$ so that no zeros nor poles of $f$ lie on $\partial P$, which is possible as $\mathbb C/\Lambda$ is compact.
    The residue theorem gives
    $$\sum_{z\text{ pole in }P}\operatorname{res}_z(f)=\frac{1}{2\pi i}\int_{\partial P}f(z)\,\mathrm dz=0$$
    by the periodicity of $f$.
    Hence either $f$ is no pole, which means $f$ is constant, or $f$ has at least two poles (counted with multiplicity), so $\deg f\ge 2$.
\end{proof}
    \section{Weierstrass' Elliptic Functions}
We shall explore the definition and properties of the Weierstrass $\wp$-functions, also known as Weierstrass' elliptic functions.
\subsection{The Definition}
We know that a non-constant elliptic function has degree at least $2$.
The Weierstrass $\wp$-function on a lattice is a elliptic function that behaves like $z\mapsto z^{-2}$ near any lattice point.
\begin{definition}
    Let $\Lambda$ be a lattice in $\mathbb C^2$.
    The associated Weierstrass $\wp$-function is defined by
    $$\wp(z)=\wp_\Lambda(z)=\frac{1}{z^2}+\sum_{\omega\in\Lambda\setminus\{0\}}\left(\frac{1}{(z-\omega)^2}-\frac{1}{\omega^2}\right)$$
\end{definition}
We have A LOT to check.
\begin{lemma}
    Let $\Lambda=\langle\omega_1,\omega_2\rangle$ be a lattice in $\mathbb C$ and $t\in\mathbb R$.
    Then the sum
    $$\sum_{\omega\in\Lambda\setminus\{0\}}\frac{1}{|\omega|^t}$$
    converges iff $t>2$.
\end{lemma}
\begin{proof}
    Consider the tilted square (or unit circle in $\ell^1$ metric) $Q=\{(t_1,t_2)\in\mathbb R^2:|t_1|+|t_2|=1\}$.
    By compactness, the continuous function $Q\to\mathbb R$ via $(t_1,t_2)\to|t_1\omega_1+t_2\omega_2|$ attains its maximum $M$ and minimum $m$ on $Q$.
    $m\neq 0$ since $\omega_1,\omega_2$ needs to be linearly independent over $\mathbb R$.
    So $0<m\le t_1\omega_1+t_2\omega_2\le M<\infty$ for any $t_1,t_2\in Q$.
    Consider $(k,l)\in\mathbb Z^2\setminus\{0\}$ and take
    $$t_1=\frac{k}{|k|+|l|},t_2=\frac{l}{|k|+|l|}$$
    Therefore $m(|k|+|l|)\le |k\omega_1+l\omega_2|\le M(|k|+|l|)$, hence the sum we wanted is bounded by positive multiples of
    $$\sum_{(k,l)\in\mathbb Z^2\setminus\{0\}}\frac{1}{(|k|+|l|)^t}$$
    So we only need to understand the convergence of this sum.
    Now for each $n\in\mathbb Z_{>0}$, the equation $n=|k|+|l|$ has exactly $4n$ solutions of $(k,l)\in\mathbb Z^2\setminus\{0\}$, therefore this sum converges iff
    $$\sum_{n=1}^\infty\frac{4n}{n^t}=4\sum_{n=1}^\infty\frac{1}{n^{t-1}}$$
    which converges iff $t>2$.
\end{proof}
\begin{theorem}
    $\wp_\Lambda$ is a well-defined elliptic function with $\Lambda$ its set of periods.
    Moreover, $\wp_\Lambda$ is even and has degree $2$.
\end{theorem}
\begin{proof}
    For convergence, we shall estimate the summands.
    \begin{align*}
        \left|\frac{1}{(z-\omega)^2}-\frac{1}{\omega^2} \right|&=\left|\frac{z(2\omega-z)}{\omega^2(z-\omega)^2} \right|\\
        &=\left|\frac{z}{\omega^2}\right|\left|\frac{2\omega-z}{(z-\omega)^2}\right|\\
        &\le \left|\frac{z}{\omega^2}\right|\left( \frac{2}{|z-\omega|}+\frac{|z|}{|z-\omega|^2} \right)
    \end{align*}
    Fix $R\ge|z|$.
    For all but finitely many $\omega$, we have $|\omega|\ge 2R$, so $|\omega-z|\ge |\omega|/2\ge R$.
    So after throwing away finitely many terms,
    $$\left|\frac{z}{\omega^2}\right|\left( \frac{2}{|z-\omega|}+\frac{|z|}{|z-\omega|^2} \right)\le\frac{R}{|\omega|^2}\left( \frac{2}{|\omega|/2}+\frac{R}{|\omega|R/2} \right)=\frac{6R}{|\omega|^3}$$
    So the sum converges by the preceding lemma since $3>2$, which means $\wp_\Lambda(z)$ is indeed well-defined and automatically meromorphic.
    It is clear that it is even.
    To see it is elliptic, choose $\omega_0\in\Lambda$, we need to show that $\omega_0$ is a period of $\wp_\Lambda$.
    Now it is clear that $\omega_0$ is a period of
    $$\wp_\Lambda^\prime(z)=\sum_{\omega\in\Lambda}\frac{-2}{(z-\omega)^3}$$
    So $f(z)=\wp_\Lambda(z+\omega_0)-\wp_\Lambda(z)$ has zero derivative, hence $f$ is constant.
    This means that $\wp_\Lambda(z+\omega_0)=\wp_\Lambda(z)+C$ for some constant $C$.
    But $\wp_\Lambda$ is even, so setting $z=-\omega_0/2$ gives $C=0$ and hence $\omega_0$ is a period, so anything in $\Lambda$ is a period of $\wp_\Lambda$.
    Also the poles of $\wp_\Lambda$ is exactly $\Lambda$, so the set of periods of $\wp_\lambda$ has to be exactly $\Lambda$.
    In particular, $\wp_\Lambda$ has a unique pole of order $2$ on $\mathbb C/\Lambda$, so $\deg\wp_\Lambda=2$
    This completes the proof.
\end{proof}
\begin{remark}
    We now know that:\\
    (i) $\wp_\Lambda$ is meromorphic with set of periods $\Lambda$.\\
    (ii) $\wp_\Lambda$ has poles only at $\Lambda$.\\
    (iii) $\wp_\Lambda(z)-z^{-2}\to 0$ as $z\to 0$.\\
    Furthermore, these properties uniquely characterised $\wp_\Lambda$ up to a constant.
\end{remark}
    \subsection{Branching Properties}
$\wp_\Lambda$ has a unique pole in $\mathbb C/\Lambda$ of order $2$.
The other ramification points are at the zeros of $\wp_\Lambda^\prime$.
Recall that
$$\wp_\Lambda^\prime(z)=\sum_{\omega\in\Lambda}\frac{-2}{(z-\omega)^3}$$
which is an odd function with degree $3$ and poles exactly at the lattice points.
For any $\omega\in\Lambda$, we have $\wp_\Lambda^\prime(\omega/2)=\wp_\Lambda^\prime(\omega/2-\omega)=\wp_\Lambda^\prime(-\omega/2)=-\wp_\Lambda^\prime(\omega/2)$ as $\wp_\Lambda^\prime$ is odd, so $\omega/2$ is either a zero or a pole.
So in the period parallelogram $P$, there are at least three zeros (up to $\Lambda$) namely $\omega_1/2,\omega_2/2,(\omega_1+\omega_2)/2$.
But $\deg\wp_\Lambda^\prime=3$, so these are all the zeros and all of them are simple.
\begin{remark}
    $\wp_\Lambda$ has $4$ ramification points in $\mathbb C/\Lambda$, namely $0,\omega_1/2,\omega_2/2,(\omega_1+\omega_2)/2$ each with multiplicity $2$.
    So by the valency theorem, they must have distinct images $\infty=\wp_\Lambda(0),e_1=\wp_\Lambda(\omega_1/2),e_2=\wp_\Lambda(\omega_2/2),e_3=\wp_\Lambda((\omega_1+\omega_2)/2)$.
\end{remark}
\begin{remark}
    By plugging in, this is consistent with Riemann Hurwitz as $\mathbb C/\Lambda$ has genus $1$.
\end{remark}
    \subsection{An Algebraic Relation}
Although $\wp_\Lambda$ is so far just an example of an elliptic function, it will be the key to classify all of them.
First, we relate $\wp_\Lambda^\prime$ to $\wp_\Lambda$ algebraically.
\begin{proposition}\label{elliptic_wp}
    There exists constant $g_2,g_3\in\mathbb C$, depending only on $\Lambda$, such that
    $$(\wp^\prime)^2=4\wp^3-g_2\wp-g_3$$
    where $\wp=\wp_\Lambda$.
\end{proposition}
\begin{proof}
    Near $0$, we have $\wp(z)=z^{-2}+az^2+o(z^4)$ for some constant $a$ as $\wp$ is even and looks like $z^{-2}$ near $0$.
    So $\wp(z)^3=z^{-6}+f(z)$ for some analytic $f$.
    Differentiating this Laurent series gives $\wp^\prime(z)=-2z^{-3}+2az+o(z^3)$, therefore $\wp^\prime(z)^2=4z^{-6}-8az^{-2}+g(z)$ for some analytic $g$.
    These would give
    $$\wp^\prime(z)^2=4\wp(z)^3-8az^{-2}-h(z)$$
    where $h$ is analytic.
    Setting $g_2=8a$ gives $(\wp^\prime)^2-4\wp^3+g_2\wp$ is analytic, has no poles, and doubly periodic, hence constant.
    Setting this constant as $-g_3$ completes the proof.
\end{proof}
These constants $g_2,g_3$ actually relates to the branch points $e_1,e_2,e_3$.
\begin{remark}
    When $z\in (1/2)\Lambda\setminus\Lambda$, we have $\wp^\prime(z)=0$ and $\wp(z)=e_i$ for some $i$.
    Then the preceding proposition means that $0=4e_i^3-g_2e_i-g_3$, so $e_1,e_2,e_3$ are exactly the three roots of $4z^3-g_2z-g_3$.
    In particular, $e_1+e_2+e_3=0$.
    Conversely, we can rewrite the relation as
    $$\wp^\prime=4(\wp-e_1)(\wp-e_2)(\wp-e_3)$$
\end{remark}
    \section{More about Weierstrass' Elliptic Functions; Quotients}
\subsection{An Elliptic Curve}
We have seen two general constructions of Riemann surfaces with genus $1$:
The complex torus and the compactification of the Riemann surface associated with $w^2=z^3-z$.
These two constructions has to be related, but how?
Here, we shall prove that any complex torus is isomorphic to an algebraic construction.
This is a corollary of Proposition \ref{elliptic_wp}.
\begin{corollary}
    Let $\mathbb C/\Lambda$ be a complex torus, then there are constants $g_2,g_3$ such that $\mathbb C/\Lambda$ is biholomorphic to a one-point compactification of the graph $X'=\{(x,y)\in\mathbb C^2:y^2=4x^3-g_2x-g_3\}$.
\end{corollary}
\begin{proof}[Sketch of proof]
    Take $g_2,g_3$ exactly as in Proposition \ref{elliptic_wp}.
    Turns out $X'$ can be compactified into a Riemann surface $X=X'\cup\{\infty\}$ with charts provided by coordinate projection.
    Define $F:\mathbb C\to X$ via $z\mapsto(\wp(z),\wp^\prime(z))$ where $\wp=\wp_\Lambda$.
    Now $\operatorname{Im}F\subset X$ by Proposition \ref{elliptic_wp} and $F$ is analytic as the charts are coordinate projections.
    So via a quotient, we can use $F$ to induce $\Phi:\mathbb C/\Lambda\to X$.
    It remains to show that $\Phi$ is a conformal equivalence.
    As it is analytic, it suffices to show that it is bijective.
    It is surjective as it is nonconstant and everything is compact.
    To see it is injective, consider the period parallelogram centered at $0$, that is the parallelogram with vertices $(\omega_1+\omega_2)/2,(\omega_2-\omega_1)/2,(-\omega_1-\omega_2)/2,(\omega_1-\omega_2)/2$.
    It suffices to show that $F$ is injective in the interior of $P$, which will imply injectivity of $\Phi$ in general due to the valency theorem.
    Suppose $F(z)=F(z')$ for $z,z'\in P^\circ$, then $\wp(z)=\wp(z')$, so $z=\pm z'$ as $\wp$ is even and has degree $2$.
    But also $\wp^\prime(z)=\wp^\prime(z')=\pm \wp^\prime(z)$ as $\wp$ is odd, so $z=z'$ as the zeros of $\wp^\prime$ are not in $P^\circ$.
    This completes the proof.
\end{proof}
    \subsection{Classification of Elliptic Functions}
We already classified the meromorphic functions on $\mathbb C_\infty$.
Remarkably, we can do something similar for the torus $\mathbb C/\Lambda$.
\begin{theorem}
    Let $f$ be an elliptic functions with periods $\Lambda$.
    There exists rational functions $Q_1,Q_2$ such that
    $$f(z)=Q_1(\wp(z))+Q_2(\wp(z))\wp^\prime(z)$$
    where $\wp=\wp_\Lambda$.
    Furthermore, $f$ is even, then $Q_2=0$.
\end{theorem}
\begin{proof}
    First assume $f$ is even.
    Now $f,\wp$ both have finitely many branch points by compactness.
    So we may choose distinct $c,d\in\mathbb C$ which are not branch points of $f,\wp$.
    Now consider the function $z\mapsto (f(z)-c)/(f(z)-d)$ which is analytic (as it is the composition of $f$ with a M\"obius transformation), even, and has simple zeros and poles not at ramification points of $\wp$.
    Note that we can invert M\"obius transformations nicely, therefore we can WLOG assume that $f$ has these properties.
    Since $f$ is even, the zeros of $f$ can be written as $\{\pm a_1,\ldots,\pm a_m\}$ where $a_i\neq\pm a_j$ for $i\neq j$.
    Likewise we can write the poles of $f$ as $\{\pm b_1,\ldots,\pm b_n\}$.
    We can write down an elliptic function with the same zeros and poles, namely
    $$g(z)=\frac{(\wp(z)-\wp(a_1))\cdots (\wp(z)-\wp(a_m))}{(\wp(z)-\wp(b_1))\cdots (\wp(z)-\wp(b_n))}$$
    So $f/g$ is elliptic but does not have any zeros or poles, therefore is constant.
    The claim follows.\\
    If $f$ is odd, then $f(z)/\wp^\prime(z)$ is even hence $f$ can be written as the desired form as well by the previous part.
    For general $f$, simply write it as the sum of odd and even parts
    $$f(z)=\frac{f(z)+f(-z)}{2}+\frac{f(z)-f(-z)}{2}$$
    which completes the proof.
\end{proof}
    \subsection{Quotients of Riemann Surfaces}
\begin{definition}
    Let a group $G$ act by homeomorphism on a space $X$.
    The action is called properly discontinuous if every compact $K\subset X$ has $\{g\in G:g(K)\cap K\neq\varnothing\}$ finite.\\
    The action is free if for every $x\in X$ the stabiliser $\operatorname{Stab}_G(x)$ is trivial.
\end{definition}
\begin{example}
    If $\Lambda$ is a lattice in $\mathbb C$, then the action of $\Lambda$ on $\mathbb C$ by translation is properly discontinuous and free.
\end{example}
\begin{lemma}
    Let $G$ be a group acting properly discontinuously and freely on a Riemann surface $R$.
    The quotient $G\backslash R$ is Hausdorff and the quotient map $\pi:R\to G\backslash R$ is a regular covering map.
\end{lemma}
\begin{proof}
    For any $p,q\in R$ with $\pi(p)\neq\pi(q)$, as $R$ is Hausdorff and locally Euclidean, we can find open $U\ni p,V\ni q$ such that $\bar{U},\bar{V}$ are compact.
    Then $K=\bar{U}\cap\bar{V}$ is also compact, so $\{g\in G:g(K)\cap K\neq\varnothing\}\supset\{g\in G:g(\bar{U})\cap\bar{V}\neq\varnothing\}$ is finite.
    Say this set is $\{g_1,\ldots,g_n\}$.
    Then for each $i$ there are disjoint open neighbourhoods $U_i\ni x$ and $V_i\ni g_iy$.
    Now $U'=U\cap\bigcap_iU_i$ and $V'=V\cap\bigcap_ig_i^{-1}(V_i)$ are disjoint open neighbourhoods of $p,q$ with $GU'$ disjoint from $GV'$.
    Hence as $\pi$ is open, $\pi(U')$ and $\pi(V')$ are the required disjoint open neighbourhoods of $\pi(p),\pi(q)$.\\
    To see $\pi$ is a regular covering map, we use a similar argument.
    For any $p\in R$, again take $U\ni p$ with compact closure $K=\bar{U}$ and let $\{1,g_1,\ldots,g_n\}$ be the set $\{g\in G:g(K)\cap K\neq\varnothing\}$.
    As the action is free, $g_ix\neq x$ for each $i$, hence there exists disjoint open neighbourhoods $U_i\ni x$ and $V_i\ni g_ix$.
    But then $U'=U\cap\bigcap_i(U_i\cap g_i^{-1}V_i)$ is evenly covered via $\pi$.
    Hence $\pi$ is a regular covering map.
\end{proof}
\begin{proposition}
    Let $R$ be a Riemann surface and let $G$ be a group acting freely and properly discontinuously by conformal equivalences on $R$.
    Then the quotient $S=G\backslash R$ can be made a Riemann surface so that the quotient map $\pi:R\to S$ is analyic and a regular covering map.
\end{proposition}
\begin{proof}
    It is obvious that $S$ is connected since $R$ is.
    The preceding lemma shows that $S$ is Hausdorff.
    The conformal structure on $S$ can be constructed analogously to what we did for the complex torus.
\end{proof}
\begin{theorem}
    Let $R$ be a compact Riemann surface of genus $g_R\ge 2$ and suppose that a group $G$ acts freely and properly discontinuously on $R$ by conformal equivalences, then $G$ is finite and $|G|\le g_R-1$.
\end{theorem}
\begin{proof}
    $S=G\backslash R$ is a Riemann surface and the quotient map $\pi:R\to S$ is an analytic covering map.
    $\deg\pi=|G|$ by construction, so in particular $G$ is finite.
    \footnote{Alternatively, just take $K=R$ in the definition of properly discontinuous action.}
    Also $\pi$ is a covering map, so it has to be unramified, therefore Riemann-Hurwitz gives $g_R-1=|G|(g_S-1)$.
    As $g_R\ge 2$, $g_S-1\ge 1$, consequently $g_R-1\ge |G|$.
\end{proof}
Note that this fails when $g_R=1$.
\begin{example}[Non-example]
    The torus $\mathbb C/\Lambda$ is a group as well, and it acts on itself by conformal equivalences via left translation $(z+\Lambda)(z_0+\Lambda)\mapsto (z+z_0)+\Lambda$.
    Easily any finite subgroup of $\mathbb C/\Lambda$ acts on $\mathbb C/\Lambda$ properly continuously through the same action, but the size of these subgroups is unbounded.
\end{example}
    \section{Uniformisation and its Consequences}
\subsection{The Uniformisation Theorem}
\begin{theorem}
    Every simply connected Riemann surfaces is conformally equivalent to either $\mathbb C_\infty$, $\mathbb C$ or $\mathbb D=D(0,1)$.
\end{theorem}
\begin{proof}
    Well beyond the scope of this course (duh!).
\end{proof}
Note that despite the existence of conformal equivalence, it might be very difficult to actually find one.
\begin{remark}
    The three items on this list are all conformally distinct.
    $\mathbb C_\infty$ is obviously distinct from the two others since it is compact.
    To see $\mathbb D$ and $\mathbb C$ are conformally distinct, just observe that there is no nonconstant analytic map $\mathbb C\to\mathbb D$ by Liouville's Theorem yet there is the (nonconstant) analytic natural inclusion $\mathbb C\hookrightarrow\mathbb C_\infty$.
\end{remark}
The uniformisation is extremely useful as it links together the topological properties and possible analytic structures of certain topological surfaces.
\begin{corollary}
    Any conformal structure on $S^2$ is conformally equivalent to $\mathbb C_\infty$.
\end{corollary}
\begin{proof}
    $S^2$ is compact and simply connected.
\end{proof}
How about surfaces with positive genus?
They are not simply connected, but we know that it has a nice universal cover (Algebraic Topology again!).
\begin{theorem}
    Every Riemann surface $R$ has a regular covering map $\pi:\tilde{R}\to R$ such that $\tilde{R}$ is simply connected.
    Furthermore, there is a group $G$ acting freely and properly discontinuously by conformal equivalences on $\tilde{R}$ and the covering map descends to a conformal equivalence $G\backslash\tilde{R}\cong R$.
\end{theorem}
\begin{proof}
    Algebraic Topology.
\end{proof}
Therefore
\begin{corollary}
    Every Riemann surface $R$ is conformally equivalent to a quotient $R\cong G\backslash\tilde{R}$ where $\tilde{R}$ is conformally equivalent to one of $\mathbb C_\infty,\mathbb C,\mathbb D$, and $G$ acts freely and properly discontinuously
\end{corollary}
\begin{proof}
    Immediate.
\end{proof}
\begin{remark}
    In fact, $G$ is just $\pi_1(R)$ acting by deck transformations.
    Or in this context, $G$ is the collection of conformal equivalences $\phi:\tilde{R}\to\tilde{R}$ such that $\pi\circ\phi=\pi$.
\end{remark}
    \subsection{Classification of Riemann Surfaces}
We roughly classified all Riemann surfaces in the preceding section by viewing them as quotients of their universal covers, which can only be one of $\mathbb C_\infty,\mathbb C,\mathbb D$.
We say the surface is uniformised by its universal cover.
For some of these cases, we can do something better.
\begin{proposition}
    If a Riemann surface $R$ is uniformised by $\mathbb C_\infty$, then $R$ is conformally equivalent to $\mathbb C_\infty$.
\end{proposition}
\begin{proof}
    Suppose $R=G\backslash\mathbb C_\infty$, then we know that $G$ acts by conformal equivalences $\mathbb C_\infty\to\mathbb C_\infty$.
    But this is just M\"obius transformations (from Example Sheet).
    However, any M\"obius transformation has at least one fixed point, but $G$ should act freely, hence necessarily $G$ is trivial and hence $R\cong\mathbb C_\infty$.
\end{proof}
What about $\mathbb C$?
\begin{proposition}
    If a Riemann surface $R$ is uniformised by $\mathbb C$, i.e. $R\cong G\backslash\mathbb C$, then one of the following holds:\\
    (i) $G$ is trivial and $R\cong\mathbb C$.\\
    (ii) $G\cong\mathbb Z$ and $R\cong\mathbb C_\star$.\\
    (iii) $G\cong\mathbb Z^2$ and $R\cong\mathbb C/\Lambda$ for a lattice $\Lambda$.
\end{proposition}
\begin{proof}
    The conformal automorphisms of $\mathbb C$ are simply the (nonconstant) linear maps $\{z\mapsto az+b:a\in\mathbb C_\star,b\in\mathbb C\}$ (again from Example Sheet).
    Which of them can $G$ act by?
    Note that if $a\neq 1$, then $z\mapsto az+b$ has a fixed point, therefore $G$ can only consist of translations.
    But then (Example Sheet again!) $G$ (identified by the values of $b$) can only be one of:\\
    (i) Trivial.\\
    (ii) $\langle\omega\rangle\cong\mathbb Z$ for some $\omega\neq 0$.\\
    (iii) a lattice $\Lambda\cong\mathbb Z^2$.\\
    And these corresponds to the three situations as stated.
\end{proof}
Oh, by the way, a surface cannot be uniformised by more than one of $\mathbb C_\infty,\mathbb C,\mathbb D$.
\begin{lemma}[Lifting Lemma]
    Let $f:R\to S$ be an analytic map of Riemann surfaces.
    Suppose $R$ is simply connected, and let $\pi:\tilde{S}\to S$ be the uniformising map of $S$, then there is an analytic map $F:R\to\tilde{S}$ such that $f=\pi\circ F$.
\end{lemma}
So we have the commutative diagram
\[
    \begin{tikzcd}
        &\tilde{S}\arrow{d}{\pi}\\
        R\arrow[dashed]{ur}{F}\arrow[swap]{r}{f}&S
    \end{tikzcd}
\]
\begin{proof}
    Example Sheet.
\end{proof}
\begin{proposition}
    A Riemann surface $R$ is uniformised by at most one of $\mathbb C_\infty,\mathbb C,\mathbb D$.
\end{proposition}
\begin{proof}
    By the previous discussion, we already know everything that is uniformised by $\mathbb C$ or $\mathbb C_\infty$ and they are distinct.
    Now suppose $R$ is uniformised by $\mathbb D$ and $\tilde{R}$ which is either $\mathbb C$ or $\mathbb C_\infty$.
    Let $\pi,f$ be the respective uniformisation maps, then by the preceding lemma, there is $F:\tilde{R}\to\mathbb D$ such that
    \[
        \begin{tikzcd}
            &\mathbb D\arrow{d}{\pi}\\
            \tilde{R}\arrow[swap]{r}{f}\arrow{ur}{F}&R
        \end{tikzcd}
    \]
    commutes.
    But then $F$ has to be constant by Liouville's Theorem, hence $f$ is also constant, contradiction.
\end{proof}
So any other Riemann surface must be uniformised by $\mathbb D$.
\begin{proposition}
    Any conformal automorphisms of $\mathbb D$ is in the form
    $$z\mapsto e^{i\theta}\frac{z-a}{1-\bar{a}z},a\in\mathbb C,\theta\in\mathbb R$$
\end{proposition}
\begin{proof}
    Complex Analysis.
\end{proof}
This is perhaps easier to picture if we send $\mathbb D$ to the open upper half plane $\mathbb H$ (by the M\"obius transformation $z\mapsto (1+iz)/(1-iz)$), which has automorphisms in the form $z\mapsto (az+b)/(cz+d)$ for $a,b,c,d\in\mathbb R,ad-bc=1$.
\begin{definition}
    A subgroup of $\operatorname{PSL}(\mathbb R)$ that acts properly discontinuously on $\mathbb H$ is called a Fuchsian group.
\end{definition}
    \subsection{Consequences of Uniformisation}
\begin{corollary}
    If $R$ is a compact Riemann surface with genus at least $2$, then it is uniformised by $\mathbb D$.
\end{corollary}
\begin{proof}
    It cannot be uniformised by $\mathbb C$ or $\mathbb C_\infty$.
\end{proof}
\begin{corollary}[Riemann Mapping Theorem]
    If $D\subsetneq\mathbb C$ is a simply connected domain, then $D$ is conformally equivalent to $\mathbb D$.
\end{corollary}
\begin{proof}
    We only have to show that $D$ is not conformally equivalent to $\mathbb C_\infty$ or $\mathbb C$.
    It certainly cannot be conformally equivalent to $\mathbb C_\infty$ since $D$ is not compact.
    Suppose $f$ is a conformal equivalence from $\mathbb C$ to $D$.
    Casorati-Weierstrass shows that the singularity of $f$ at $\infty$ is not essential (as $f$ has to be injective).
    So $\infty$ is either removable or a pole, therefore $f$ extends to $\bar{f}:\mathbb C_\infty\to D\cup\{\bar{f}(\infty)\}\subset\mathbb C_\infty$.
    But now $\mathbb C_\infty$ is compact, so $\bar{f}:\mathbb C_\infty\to\mathbb C_\infty$ is surjective, hence $\bar{f}(\infty)=\infty$ and thus $D=\mathbb C$, contradiction.
\end{proof}
\begin{corollary}[Picard's Theorem]
    Any analytic function $\mathbb C\to\mathbb C\setminus\{0,1\}$ is constant.
\end{corollary}
Of course we can replace $\{0,1\}$ by any two distinct points in $\mathbb C$.
\begin{proof}
    $\mathbb C\setminus\{0,1\}$ is uniformised by $\mathbb D$ by Example Sheet.
    The statement then follows from the lifting lemma.
\end{proof}
\end{document}