\subsection{Branching Properties}
$\wp_\Lambda$ has a unique pole in $\mathbb C/\Lambda$ of order $2$.
The other ramification points are at the zeros of $\wp_\Lambda^\prime$.
Recall that
$$\wp_\Lambda^\prime(z)=\sum_{\omega\in\Lambda}\frac{-2}{(z-\omega)^3}$$
which is an odd function with degree $3$ and poles exactly at the lattice points.
For any $\omega\in\Lambda$, we have $\wp_\Lambda^\prime(\omega/2)=\wp_\Lambda^\prime(\omega/2-\omega)=\wp_\Lambda^\prime(-\omega/2)=-\wp_\Lambda^\prime(\omega/2)$ as $\wp_\Lambda^\prime$ is odd, so $\omega/2$ is either a zero or a pole.
So in the period parallelogram $P$, there are at least three zeros (up to $\Lambda$) namely $\omega_1/2,\omega_2/2,(\omega_1+\omega_2)/2$.
But $\deg\wp_\Lambda^\prime=3$, so these are all the zeros and all of them are simple.
\begin{remark}
    $\wp_\Lambda$ has $4$ ramification points in $\mathbb C/\Lambda$, namely $0,\omega_1/2,\omega_2/2,(\omega_1+\omega_2)/2$ each with multiplicity $2$.
    So by the valency theorem, they must have distinct images $\infty=\wp_\Lambda(0),e_1=\wp_\Lambda(\omega_1/2),e_2=\wp_\Lambda(\omega_2/2),e_3=\wp_\Lambda((\omega_1+\omega_2)/2)$.
\end{remark}
\begin{remark}
    By plugging in, this is consistent with Riemann Hurwitz as $\mathbb C/\Lambda$ has genus $1$.
\end{remark}