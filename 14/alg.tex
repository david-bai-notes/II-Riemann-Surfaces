\subsection{An Algebraic Relation}
Although $\wp_\Lambda$ is so far just an example of an elliptic function, it will be the key to classify all of them.
First, we relate $\wp_\Lambda^\prime$ to $\wp_\Lambda$ algebraically.
\begin{proposition}\label{elliptic_wp}
    There exists constant $g_2,g_3\in\mathbb C$, depending only on $\Lambda$, such that
    $$(\wp^\prime)^2=4\wp^3-g_2\wp-g_3$$
    where $\wp=\wp_\Lambda$.
\end{proposition}
\begin{proof}
    Near $0$, we have $\wp(z)=z^{-2}+az^2+o(z^4)$ for some constant $a$ as $\wp$ is even and looks like $z^{-2}$ near $0$.
    So $\wp(z)^3=z^{-6}+f(z)$ for some analytic $f$.
    Differentiating this Laurent series gives $\wp^\prime(z)=-2z^{-3}+2az+o(z^3)$, therefore $\wp^\prime(z)^2=4z^{-6}-8az^{-2}+g(z)$ for some analytic $g$.
    These would give
    $$\wp^\prime(z)^2=4\wp(z)^3-8az^{-2}-h(z)$$
    where $h$ is analytic.
    Setting $g_2=8a$ gives $(\wp^\prime)^2-4\wp^3+g_2\wp$ is analytic, has no poles, and doubly periodic, hence constant.
    Setting this constant as $-g_3$ completes the proof.
\end{proof}
These constants $g_2,g_3$ actually relates to the branch points $e_1,e_2,e_3$.
\begin{remark}
    When $z\in (1/2)\Lambda\setminus\Lambda$, we have $\wp^\prime(z)=0$ and $\wp(z)=e_i$ for some $i$.
    Then the preceding proposition means that $0=4e_i^3-g_2e_i-g_3$, so $e_1,e_2,e_3$ are exactly the three roots of $4z^3-g_2z-g_3$.
    In particular, $e_1+e_2+e_3=0$.
    Conversely, we can rewrite the relation as
    $$\wp^\prime=4(\wp-e_1)(\wp-e_2)(\wp-e_3)$$
\end{remark}