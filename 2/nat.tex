\section{Natural Boundary; A Gluing Construction; Roots}
\subsection{Natural Boundary}
Sometimes, it is impossible to do analytic continuation.
Write $\mathbb D=D(0,1)$ and $\mathbb T=\partial\mathbb D=S^1$.
Consider a power series $f(z)=\sum_na_nz^n$ with radius of convergence $1$.\\
We say $z_0\in\mathbb T$ is regular if there is a neighbourhood $U$ of $z_0$ such that there exists a function element $(g,U)$ such that $f|_{U\cap\mathbb D}=g|_{U\cap\mathbb D}$.
A point that is not regular is singular.
Easily, the set of regular points is open in $\mathbb T$, so the set of singular points has to be closed.
There are something to beware of which are illustrated in the example below.
\begin{example}
    Consider $f(z)=\frac{1}{1-z}=\sum_{n\ge 0}z^n$.
    We know that the set of singular point is just $\{1\}$.
    However, the power series evaluated at $-1$ does not converge.
    So a regular point needs not guarantee that the power series converge there.
\end{example}
\begin{example}
    Consider the power series
    $$g(z)=\sum_{n\ge 2}\frac{z^n}{n(n-1)}$$
    Then the series converges at $z=1$, but $1$ is not regular for $g$ because it is singular for $f=g^{\prime\prime}$.
\end{example}
The moral od the examples is whether or not a point is regular does not relate directly to whether or not the power series converges there.
\begin{proposition}
    If a power series $f(z)=\sum_{n\ge 0}a_nz^n$ has radius of convergence $1$, then some point of $\mathbb T$ is singular.
\end{proposition}
\begin{proof}
    Just use the compactness of $\mathbb T$.
\end{proof}
\begin{definition}
    If every point of $\mathbb T$ is singular, we say $\mathbb T$ is the natural boundary of $f$.
\end{definition}
\begin{remark}
    We can extend this definition to other simple curves in $\mathbb C_\infty$.
\end{remark}
\begin{example}
    Consider the function $f(z)=\sum_{n\ge 0}z^{n!}$.
    We shall show that $\mathbb T$ is its natural boundary.
    It suffices to show any point in the form $\omega=e^{2\pi ip/q}$ is singular, where $p,q\in\mathbb Z,q\neq 0$.
    Indeed, fix such an $\omega$ with corresponding $p,q$, then for $r\in(0,1)$,
    $$f(r\omega)=\sum_{n=0}^{q-1}r^{n!}\omega^{n!}+\sum_{n\ge q}r^{n!}$$
    But the second term definitely converge.
    To see this, we observe that for any $M$,
    $$\lim_{r\to 1}\sum_{n=q}^{M+q}r^{n!}=M+1\implies \sum_{n=q}^{M+q}r^{n!}>M$$
    for $r$ sufficiently close to $1$.
    So the second term converges.
    Yet the first term is bounded, therefore $f(r\omega)\to\infty$ as $r\to1$, so $\omega$ has to be singular at $\omega$.
\end{example}