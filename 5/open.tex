\subsection{The Open Mapping Theorem}
We want to study complex analytic functions $f:R\to\mathbb C$ where $R$ is a Riemann surface.
The case where $R$ is compact is especially interesting as we can greatly constrain them using the open mapping theorem.
\begin{theorem}\label{open_mapping}
    Any non-constant, analytic map of Riemann surfaces $f:R\to S$ is an open map.
\end{theorem}
\begin{proof}
    By the identity principle for Riemann surfaces shows that $f$ is not constant in any open subset.
    Let $W\subset R$ be open and $p\in W$.
    Pick charts $(\phi,U)$ containing $p$ and $(\psi,V)$ containing $f(p)$, then by the open mapping theorem on complex plane, $\psi\circ f(U\cap W\cap f^{-1}(V))$ is an open neighbourhood of $\psi\circ f(p)$ in $\psi(f(W)\cap V)$, so $f(U\cap W\cap f^{-1}(V))$ is an open neighbourhood of $f(p)$ in $f(W)$.
\end{proof}
\begin{corollary}
    Let $f:R\to S$ be a non-constant, analytic map of Riemann surfaces.
    If $R$ is compact, then $f$ is surjective and $S$ is also compact.
\end{corollary}
\begin{proof}
    By Theorem \ref{open_mapping}, $f(R)$ is open.
    But $R$ is compact, so $f(R)$ is compact, hence $f(R)$ is also closed as $S$ is Hausdorff.
    But then $S$ is path-connected hence connected, therefore $S=f(R)$ and hence is compact.
\end{proof}
\begin{corollary}
    Every analytic function on a compact Riemann surface is constant.
\end{corollary}
\begin{proof}
    $\mathbb C$ is not compact.
\end{proof}