\section{Complex Tori; the Open Mapping Theorem}
\subsection{Complex Tori}
So far, we only know one compact Riemann surface, namely the Riemann sphere $\mathbb C_\infty$.
One can also picture a torus being compact and can admit a conformal structure.
We shall formalise such constructions.\\
Let $\tau_1,\tau_2\in\mathbb C_\star$ such that they are linearly independent over $\mathbb R$.
Let $\Lambda$ be the additive subgroup generated by $\tau_1,\tau_2$ which is a lattice.
We then define the torus as the quotient group $T=\mathbb C/\Lambda$ that inherits the quotient topology.
We can study this topology via the fundamental parallelogram $P$ with vertices $0,\tau_1,\tau_2,\tau_1+\tau_2$.
With a little geometrical intuition we are hinted that $T\cong S^1\times S^1$ since we are just gluing the sides of the fundamental parallelogram as topological spaces which is easy enough to check.
Consequently it is compact.\\
Now the quotient map $\pi:\mathbb C\to T=\mathbb C/\Lambda$ is a regular covering map.
To prove this, take $0<\epsilon<\min\{|\lambda|:\lambda\in\Lambda\setminus\{0\}\}/2$ which one can verify is well-defined.
Then
$$\pi^{-1}(\pi(D(z_0,\epsilon)))=\bigcup_{\lambda\in\Lambda}(D(z_0,\epsilon)+\lambda)=\coprod_{\lambda\in\Lambda}(D(z_0,\epsilon)+\lambda)\cong D(z_0,\epsilon)\times\Lambda$$
by our definition of $\epsilon$.
Here $\Lambda$ has the discrete topology it inherited as a subspace of $\mathbb C$.
So $\pi$ is indeed a regular covering map.
Now, we use $\pi$ to construct an atlas on $T$.
For $p=z_0+\Lambda\in T$, let $U=\pi(D(z_0,\epsilon))$ for $\epsilon>0$ as before and $(\phi,U)$ is a chart where $\phi=(\pi|_{D(z_0,\epsilon)})^{-1}$.
This works since $\pi$ is a regular covering map.
Now for any other chart constructed in this way, i.e. $(\psi,V)=((\pi|_{D(z_1,\epsilon)})^{-1},\pi(D(z_1,\epsilon)))$, then $U\cap V$ is nonempty iff there is some $\lambda\in\Lambda$ (necessarily unique because of the bound on $\epsilon$) such that $|z_0-(z_1+\lambda)|<2\epsilon$.
In this case, the transition function is just $z\mapsto z+\lambda$ which is analytic.
So this is indeed an atlas.
This extends to a conformal structure on $T$ that makes it a Riemann surface.
It is easy to see that all of these tori are homeomorphic as they are all homeomorphic to $S^1\times S^1$.
But (as will be proven in example sheets) there are infinitely many conformal equivalence classes among these tori.