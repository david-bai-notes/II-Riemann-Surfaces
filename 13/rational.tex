\section{Rational and Periodic Functions}
\subsection{Rational Functions}
We want to study meromorphic functions in the Riemann sphere, which are simply the rational functions.
\begin{proposition}
    Every meromorphic function $f:\mathbb C_\infty\to\mathbb C_\infty$ on the Riemann sphere is a rational function, that is we can write
    $$f(z)=c\frac{(z-a_1)\cdots (z-a_m)}{(z-b_1)\cdots (z-b_n)}$$
    for $c,a_i,b_i\in\mathbb C$.
\end{proposition}
Given an $f$ in this form, we can assume WLOG that $a_i\neq b_j$ for all $i,j$.
\begin{proof}
    The case that $f$ is constant is trivial.
    Replacing $f\to 1/f$ necessary, then we can assume WLOG that $f(\infty)\neq\infty$.
    Let $b_1,\ldots,b_{n'}$ be poles of $f$ (there are finitely many as the Riemann sphere is compact).
    Then by assumption $b_i\neq\infty$ for all $i$.
    The Laurent series about $b_j$ then has the form
    $$f(z)=\sum_{l=-k_j}^\infty c_{j,l}(z-b_j)^l,c_{-k_j,j}\neq 0$$
    then $k_j$ is the order of the pole.
    Let
    $$Q_j(z)=\sum_{l=-k_j}^{-1} c_{j,l}(z-b_j)^l$$
    be the principal part of the series and let
    $$g(z)=f(z)-\sum_{j=1}^{n'}Q_j(z)$$
    Then $g$ is entire and has a removable singularity at $\infty$, hence it has to be constant, which means $f$ has to be rational.
\end{proof}
\begin{remark}
    In th proof, $f(\infty)\in\mathbb C$ means $m\le n$, and in this case we have $\deg f=\sum_jk_j=n$, therefore in general the degree of a rational function is $\max\{m,n\}$.
\end{remark}